% arara: biber
% arara: pdflatex: {options: "-file-line-error-style"}
%        File: dissertation.tex
%     Created: Thu Feb 02 01:00 AM 2023 G
% Last Change: Thu Feb 02 01:00 AM 2023 G
%
\documentclass[a4paper, 12pt]{article}

\newcommand{\titlename}{Untangling Knots Through Curve Repulsion}

\linespread{1.0}

\usepackage[backend=biber]{biblatex}
\addbibresource{bibliography.bib}

\usepackage{lmodern}
\usepackage{draftwatermark}
%\SetWatermarkScale{5}

\usepackage[textheight=22.5cm, textwidth=15cm]{geometry}
\usepackage[]{graphicx}
\usepackage{lipsum}
\usepackage{chngcntr}
%\counterwithin{section}{part}
%\renewcommand{\thepart}{\arabic{part}}
\usepackage[toc,page]{appendix}
\usepackage{caption}
\usepackage[labelformat=simple]{subcaption}
\renewcommand\thesubfigure{(\alph{subfigure})}
\usepackage[]{amsmath}
\numberwithin{equation}{section}
\numberwithin{figure}{section}
\usepackage{amssymb}
\usepackage{amsthm}
\theoremstyle{definition}
\newtheorem*{definition}{Definition}
\theoremstyle{plain}
\newtheorem{theorem}{Theorem}
\theoremstyle{plain}
\newtheorem{lemma}[theorem]{Lemma}
\theoremstyle{plain}
\newtheorem{corollary}[theorem]{Corollary}
\theoremstyle{remark}
\newtheorem*{remark}{Remark}
\numberwithin{theorem}{section}
\theoremstyle{definition}
\newtheorem*{example}{Example}
\renewcommand\qedsymbol{$\blacksquare$}
\renewcommand\labelitemii{$\circ$}
\usepackage{mathtools}
\usepackage{accents}
\usepackage{algpseudocode}
\usepackage[section]{algorithm}
\usepackage{subfiles}               % BEST LOADED AT THE END OF PREAMBLE
\newcommand{\xbf}{\mathbf{x}}
\newcommand{\ybf}{\mathbf{y}}
\newcommand{\zbf}{\mathbf{z}}
\newcommand{\abf}{\mathbf{a}}
\newcommand{\cbf}{\mathbf{c}}
\newcommand{\pbf}{\mathbf{p}}
\newcommand{\qbf}{\mathbf{q}}
\newcommand{\ubf}{\mathbf{u}}
\newcommand{\ebf}{\mathbf{e}}
\newcommand{\bbf}{\mathbf{b}}
\newcommand{\nbf}{\mathbf{n}}
\newcommand{\Tbf}{\mathbf{T}}
\newcommand{\zerobf}{\mathbf{0}}
\newcommand{\xibf}{\boldsymbol{\xi}}
\newcommand{\laplacian}{\Delta}
\newcommand{\curvenabla}{\tilde{\nabla}}
\newcommand{\curvelaplacian}{\tilde{\laplacian}}
\newcommand{\gammabf}{\boldsymbol{\gamma}}
\newcommand{\Gammabf}{\boldsymbol{\Gamma}}
\newcommand{\inner}[1]{\langle #1 \rangle}
\newcommand{\norm}[1]{\lvert \lvert #1 \rvert \rvert}
\newcommand{\intd}{\, \text{d}}
\newcommand{\idf}{\, \text{Id}}
\newcommand{\suchthat}{\text{\,s.t.\,}}
\newcommand{\candidatenumber}{(Redacted)}
\DeclareMathOperator{\grad}{grad}
\DeclareMathOperator{\Grad}{Grad}
\DeclarePairedDelimiter{\abs}{\lvert}{\lvert}
\def\Xint#1{\mathchoice
   {\XXint\displaystyle\textstyle{#1}}%
   {\XXint\textstyle\scriptstyle{#1}}%
   {\XXint\scriptstyle\scriptscriptstyle{#1}}%
   {\XXint\scriptscriptstyle\scriptscriptstyle{#1}}%
   \!\int}
\def\XXint#1#2#3{{\setbox0=\hbox{$#1{#2#3}{\int}$}
     \vcenter{\hbox{$#2#3$}}\kern-.5\wd0}}
\def\ddashint{\Xint=}
\def\dashint{\Xint-}

\begin{document}

% STRUCTURE
% Introduction
% Gradient Flow Equation
% Finite Difference Method
% Unknotting Curves Via Energy Minimisation with Gradient Flow Equation
% Simpler Unknotting via Approximation Theory
% Bibliography

\subfile{sections/cover}
\subfile{sections/abstract}
\tableofcontents

\section{Introduction}
\subfile{sections/introduction}

\section{Gradient Flow Equation}
Since we pose the problem as continuous reduction of some functional,
we need an applicable framework.
In our case, \textbf{gradient flow equation} seems to be appropriate.
\subsection{Motivation of Gradient Flow Equation}
\label{sct: Motivation of Gradient Flow Equation}
\subfile{sections/gradientFlow}
\subsection{Gradient of Functional}
\subfile{sections/functionalGradient}

\section{Finite Difference Method}
One of the simplest method to solve a differential equation is the \textbf{finite difference method}.\cite{nspde}
\subfile{sections/finiteDifference}

\section{Tangent-Point Energy}
\subfile{sections/tangentPointEnergy}

\section{Unknotting Curves via Gradient Flow Equation}
\subfile{sections/unknottingCurve}

\section{Conclusion, Outlook, and Potential Alternative Approach}
\subfile{sections/conclusion}

\subsection{An Alternative Approach via Approximation Theory}
\subfile{sections/FourierSeries}

\subsection{Potential Extension: Ribbon-like Objects}
\subfile{sections/Membrane}

\newpage
\begin{appendices}
    \section{Definitions of Important Inner Product Spaces}
    \label{sct: Definitions of Important Inner Product Spaces}
    \subfile{sections/innerProductSpaces}
    \section{Gradient Operator in Sobolev Spaces}
    \label{sct: Integration By Parts}
    \subfile{sections/ibp}
    \section{Tangent-Point Energy Quadrature: Other Homeomorphism Classes}
    \label{sct: Other Homeomorphism Classes}
    \subfile{sections/otherHomeomorphism}
    \section{Exact $\ell^2$ Gradient of $E_{\beta}^{\alpha}$}
    \label{sct: Exact Gradient}
    \subfile{sections/analyticalDerivativeOfTangentPointEnergy}
\end{appendices}

\printbibliography
\end{document}


