\documentclass[../dissertation.tex]{subfiles}
\begin{document}
The dissertation explored a few methods of untangling a curve.
In practice, however, we face untangling problem with ribbon-like objects as well,
which hold additional properties such as intrinsic curvatures, surface tension, and elasticity.

One could model this as a membrane $\Sigma$ and consider additional energy such as following\cite{mmb}:
\begin{equation}
    \mathcal{E}_{\Sigma} \coloneqq \iint_{\Sigma} \left( \gamma + 2 \kappa \left( H-H_0 \right)^2 + \bar \kappa K_g \right) \intd S
\end{equation}
where
\begin{itemize}
    \item $H$ and $K_G$ are the mean and Gaussian curvatures respectively.
    \item $\gamma$ is the surface tension.
    \item $\kappa$ and $\bar \kappa$ are the bending and saddle-splay modulus respectively.
    \item $H_0$ is the intrinsic mean curvature.
\end{itemize}
By reducing $\mathcal{E}_{\beta}^{\alpha} + \mathcal{E}_{\Sigma} + \mathcal{C}$, one could have more control untangling objects such as DNA strands.

\end{document}
