\documentclass[../dissertation.tex]{subfiles}

\begin{document}

\section{Tangent-Point Energy}
Given a curve $\gammabf:M \rightarrow \mathbb{R}^3$,
a sensible choice of curve energy is the tangent-point energy\cite{YSC2021}.
\begin{definition}
    For a continuously differentiable parameterised curve $\gammabf:M \rightarrow \mathbb{R}^3$, define \textbf{tangent-point energy} as:
    \begin{equation}
        \mathcal{E}_{\beta}^{\alpha} \left( \gammabf \right) \coloneqq \iint_{M^2} k_{\beta}^{\alpha} (\gammabf_x, \gammabf_y, \Tbf_x) \intd \gamma_x \intd \gamma_y 
        \label{equ: Tangent-Point Energy}
    \end{equation}
    where $\Tbf_x \coloneqq \left. \frac{\intd \gammabf_x}{\intd t} \middle/ \abs{\frac{\intd \gammabf_x}{\intd t}} \right. $ is the unit tangent vector at $\gammabf_x$ along the curve,
    and \textbf{tangent-point kernel} is given as:
    \begin{equation}
        k_{\beta}^{\alpha} \left( \pbf, \qbf, \Tbf \right) \coloneqq \frac{\abs{\Tbf \wedge \left( \pbf - \qbf \right)}^{\alpha}}{\abs{\pbf - \qbf}^{\beta}}
    \end{equation}
    $\alpha$ and $\beta$ are parameters one could choose, but for tangent-point energy to be well-defined,
    one may choose them to satisfy $\alpha >1$ and $\beta \in \left[ \alpha+2, 2\alpha + 1 \right)$.

    Note that choosing $\alpha = 2$ and $\beta = 4$ results in the original tangent-point energy by Buck and Orloff\cite{BO1995}.
    %For the rest of the dissertation, unless specified, we use $\mathcal{E}_{4}^{2}$.
\end{definition}

The choice of parameters $\alpha$ and $\beta$ changes the behaviour of the tangent-point energy as stated in the following lemma.
\begin{lemma}
    Tangent-point energy $\mathcal{E}_{\beta}^{\alpha}$ defined as (\ref{equ: Tangent-Point Energy}) is scale invariant with respect to the curve if and only if $\beta = \alpha + 2$.
    Moreover, if $\beta > \alpha + 2$, then $\mathcal{E}_{\beta}^{\alpha}$ scales inversely with the curve.
\end{lemma}
\begin{proof}
    Take a parameterised curve $\gammabf:M \rightarrow \mathbb{R}^3$ and $\Gammabf \coloneqq c \gammabf$, a curve scaled by factor $c>0$ of $\gammabf$.
    Note that the unit tangent vector is identical for $\gammabf$ and $\Gammabf$, that is, $\Tbf_x \coloneqq \left. \frac{\intd \gammabf_x}{\intd t} \middle/ \abs{\frac{\intd \gammabf_x}{\intd t}} \right. = \left. \frac{\intd \Gammabf_x}{\intd t} \middle/ \abs{\frac{\intd \Gammabf_x}{\intd t}} \right. $

    Then,
    \begin{equation}
        \frac{k_{\beta}^{\alpha} \left( \gammabf_x, \gammabf_y, \Tbf_x \right)}{k_{\beta}^{\alpha} \left( \Gammabf_x, \Gammabf_y, \Tbf_x \right)}
        =
        \frac{\abs{\gammabf_x - \gammabf_y}^{\alpha}}{\abs{\Gammabf_x - \Gammabf_y}^{\beta}} = c^{\beta - \alpha}
    \end{equation}
    Also note that
    \begin{equation}
        \intd \Gamma = c \intd \gamma
    \end{equation}
\end{proof}

\end{document}
