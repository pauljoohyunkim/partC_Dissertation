\documentclass[../dissertation.tex]{subfiles}
\begin{document}
Now that gradient flow equation and tangent-point energy are introduced,
one can formalise the process of untangling a tangled curve:

\begin{definition}[Curve Untangling Process]
    Given a parameterised curve $\gammabf:M \times T \rightarrow \mathbb{R}^3$ over an interval $M$ and time domain $T$,
    denote the following initial value problem as \textbf{curve untangling process}:
    \begin{align}
        \frac{\partial \gammabf}{\partial t} &= - \grad_{\mathcal{X}} \mathcal{E}_{\beta}^{\alpha} (\gammabf) - \grad_{\mathcal{X}} \mathcal{C} (\gammabf) 
        \label{equ: Curve Untangling Process}
        \\
        \gammabf(s;0) &= \gammabf_0 (s)
        \label{equ: Initial Curve}
    \end{align}
    where 
    \begin{itemize}
        \item $\gammabf_0 (s)$ is the parameterisation of the initial (tangled) curve (prescribed at $t=0$)
        \item $\mathcal{E}_{\beta}^{\alpha}$ is the tangent-point energy (See (\ref{equ: Tangent-Point Energy}))
        \item $\mathcal{C}$ is additional constraint energy to control behaviour of curve untangling process.
    \end{itemize}
\end{definition}

%\subsection{Constraint Energy}
%From lemma \ref{lemma: Scale-Invariance}, in the case that $\beta > \alpha + 2$,
%one could trivially minimise tangent-point energy $\mathcal{E}_{\beta}^{\alpha}$ of the curve by scaling the curve to infinity.
%In order to avoid this phenomenon, or to change the behaviour of the flow, one may add additional energy which penalises unwanted behaviours.
%
%Here are some examples of constraint one could take.
%\begin{itemize}
%    \item Taking $\mathcal{C} \coloneqq \lambda \int_{M} \norm{\gammabf}^{p} \intd \gamma$ adds the motivation for the curve not to stray away from the origin.
%    \item Taking $\mathcal{C} \coloneqq \lambda \left( \int_M \intd \gamma - L \right)^p$ adds the motivation for the curve to stay close to some constant arc-length $L$.
%\end{itemize}
%For both cases,
%$\lambda > 0$ is the strength of the overall constraint, and
%$p > 0$ is the order of local strength, where higher $p$ would lead to harsher ``thresholding''.


\subsection{Discretisation for Numerical Computation}
Solving (\ref{equ: Curve Untangling Process}), (\ref{equ: Initial Curve}) analytically is challenging.
Rather, we aim to acquire a numerical solution.
Assume for simplicity that the curve of interest is simple closed.\footnote{
    We may assume that the function definition of $\gammabf:M \rightarrow \mathbb{R}^3$ extends to $\gammabf: \mathbb{R} \rightarrow \mathbb{R}^3$ by periodicity.
}

\subsubsection{Discretisation of Curve}
We start by discretising the initial curve $\gammabf_0$ by taking $N$ points on a curve as shown in Figure \ref{fig: Discretization of Curve}.
Represent the initially discretised curve as $\Gammabf^0 = \left( \xbf_0^0, \xbf_1^0, \cdots, \xbf_{N-1}^0 \right)$,
and the discretised curve at time step $k \in \mathbb{N} \cup \left\{ 0 \right\}$ as $\Gammabf^k = \left( \xbf_0^k, \xbf_1^k, \cdots, \xbf_{N-1}^k \right)$.
Since we restrict our attention to a simple closed curve, it is convenient to extend the indexing rule by:
\begin{equation}
    \xbf_{i}^k = \xbf_{r\left( i,N \right)}^k \hspace{1cm} \text{where } r\left( i,N \right) = \text{(remainder of } i \div N \text{)}
\end{equation}
so that $\xbf_N^k = \xbf_{0}^k$, $\xbf_{N+1}^k = \xbf_{1}^k$, etc.

Assign a function definition $\Gammabf^k: \mathbb{Z} \rightarrow \left\{ \xbf_0^k, \xbf_1^k, \cdots, \xbf_{N-1}^k \right\}$ as:
\begin{equation}
    \Gammabf^k (i) = \xbf_{r\left( i, N \right)}^k 
\end{equation}
anaogous to $\gammabf = \gammabf(s)$ being a parameterised curve, which is a vector-valued function.

Finally, denote by $e_{i}^k$ for the (undirected) edge with vertex pair $\left( \xbf_i^k, \xbf_{i+1}^k \right)$.

\subsubsection{Discretisation of Tangent-Point Energy}
In order to acquire numerical solution, one must also be able to numerically compute the tangent-point energy.
For this, we pose the energy quadrature $E_{\beta}^{\alpha} \left( \Gammabf \right)$ of the following form:
\begin{equation}
    \mathcal{E}_{\beta}^{\alpha} \left( \gammabf \left( \cdot, t \right) \right) \coloneqq
    \iint_{M^2} k_{\beta}^{\alpha} (\gammabf_x, \gammabf_y) \intd \gamma_x \intd \gamma_y
    \approx
    E_{\beta}^{\alpha} \left( \Gammabf^k \right) \coloneqq
    \sum_{i, j \in \left\{ 0, \cdots, N-1 \right\}} K_{\beta}^{\alpha} \left( i, j \right) \norm{e_i^k} \, \norm{e_j^k}
    \label{equ: Energy Approximation Form}
\end{equation}
where
$K_{\beta}^{\alpha}$ is an approximation of tangent-point kernel $k_{\beta}^{\alpha}$ (which is specified at (\ref{equ: Kernel 4-point})).
Note that $\Gammabf^k$ is a polygonal curve, for which tangent-point energy (\ref{equ: Tangent-Point Energy}) is not well-defined due to locally non-integrable contributions from vertices:

\begin{figure}[tbp]
    \centering
    \includegraphics[width=0.8\textwidth]{sections/discretizationImgs/discretization}
    \caption{Discretisation of a simple closed curve by sampling the points along the curve.}
    \label{fig: Discretization of Curve}
\end{figure}

One way to resolve the issue is to ``ignore'' the adjacent edge contribution\cite{YSC2021} in the energy quadrature $E_{\beta}^{\alpha}$ as shown in Figure \ref{fig: Energy discretization by ignoring adjacent edges}.
The justification is that as we take a finer mesh ($N$ sufficiently large), the product of edge lengths ($\norm{e_i}\norm{e_j}$) should tend to zero sufficiently fast, resulting in approximation of the energy for the smooth curve, which did not have vertices resulting in local non-integrability in the first place.
\begin{figure}[tbp]
    \centering
    \begin{subfigure}[b]{0.75\textwidth}
        \centering
        \includegraphics[width=\textwidth]{sections/unknottingCurveImgs/energyDiscretization1}
        \caption{For a chosen edge $e_i^k$, ignore the two adjacent edges $e_{i-1}^k$, $e_{i+1}^k$.
            In the limit as $N \rightarrow 0$, because the edge lengths tend to zero,
            the discrepancy between the quadrature and the analytical value of the energy is expected to tend to zero.
                }
        \label{fig: Energy discretization by ignoring adjacent edges}
    \end{subfigure}
    \par\bigskip
    \begin{subfigure}[b]{0.75\textwidth}
        \centering
        \includegraphics[width=\textwidth]{sections/unknottingCurveImgs/energyDiscretization2}
        %\caption{Tangent-point kernel is approximated by 4-point quadrature $K_\beta^\alpha (i, j)$ defined as (\ref{equ: Kernel 4-point}).}
        \caption{Tangent-point kernel is approximated by 4-point quadrature defined as (\ref{equ: Kernel 4-point}).}
        \label{fig: Tangent-point kernel approximation}
    \end{subfigure}
    \caption{Quadrature for approximation of tangent-point energy.}
\end{figure}

It still remains to sensibly approximate the kernel $k_{\beta}^{\alpha} (\gammabf_x, \gammabf_y) \approx K_{\beta}^{\alpha} (i, j)$.
One sensible approximation is to use the following 4-point quadrature\cite{YSC2021} (See Figure \ref{fig: Tangent-point kernel approximation}):
\begin{multline}
    K_{\beta}^{\alpha} (i, j) \coloneqq \frac{1}{4} \biggl( k_{\beta}^{\alpha} \left( \xbf_i^k, \xbf_j^k, \Tbf_i^k \right)
        + k_{\beta}^{\alpha} \left( \xbf_i^k, \xbf_{j+1}^k, \Tbf_i^k \right) \\
        + k_{\beta}^{\alpha} \left( \xbf_{i+1}^k, \xbf_j^k, \Tbf_i^k \right)
        + k_{\beta}^{\alpha} \left( \xbf_{i+1}^k, \xbf_{j+1}^k, \Tbf_i^k \right)
    \biggr)
    \label{equ: Kernel 4-point}
\end{multline}
where $\Tbf_{i}^k \coloneqq \frac{\xbf_{i+1}^k - \xbf_i^k}{\norm{\xbf_{i+1}^k - \xbf_i^k}}$ approximates the tangent vector to the curve at $\gammabf_x$.

Putting (\ref{equ: Energy Approximation Form}) and (\ref{equ: Kernel 4-point}) together,
one can write the \textbf{tangent-point energy quadrature} as:
\begin{equation}
    E_{\beta}^{\alpha} \coloneqq \sum_{\substack{i, j \in \left\{ 0, \cdots, N-1 \right\} \\ r \left( i-j,N \right) > 1}} K_{\beta}^{\alpha} (i,j) \norm{e_i^k} \, \norm{e_j^k}
    \label{equ: Tangent-Point Energy Quadrature}
\end{equation}
where $r\left( i-j,N \right)$ is the geodesic distance between $i$ and $j$ in modulo $N$,
characterising the avoidance of adjacent edges on simple closed polygonal curve.



\subsubsection{Finite Difference Scheme of Curve Untangling Process}
Based on (\ref{equ: Curve Untangling Process}), one writes the following finite difference scheme:
\begin{equation}
    \mathcal{D}_{t} \Gammabf^{k} = -\grad_{\mathcal{X}} E_{\beta}^{\alpha} (\Gammabf^k) - \grad_{\mathcal{X}} C \left( \Gammabf^k \right) \hspace{1cm} \text{for } k=0,1,\cdots
\end{equation}
where $\mathcal{D}_t$ is the finite difference operator over time step.
For the simplest scheme, one could take the forward difference operator as $\mathcal{D}_t \Gammabf^k(i) \coloneqq \frac{\Gammabf^{k+1}(i) - \Gammabf^{k}(i)}{\Delta T}$.

\subsection{$L^2$ Explicit Euler Scheme}

\end{document}
