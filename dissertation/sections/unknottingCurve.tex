\documentclass[../dissertation.tex]{subfiles}
\begin{document}
Now that gradient flow equation and tangent-point energy are introduced,
one can formalise the process of untangling a tangled curve:

\begin{definition}[Curve Untangling Process]
    Given a parameterised curve $\gammabf:M \times T \rightarrow \mathbb{R}^3$ over an interval $M$ and time domain $T$,
    denote the following initial value problem as \textbf{curve untangling process}:
    \begin{align}
        \frac{\partial \gammabf}{\partial t} &= - \grad_{\mathcal{X}} \mathcal{E}_{\beta}^{\alpha} (\gammabf) - \grad_{\mathcal{X}} \mathcal{C} (\gammabf) 
        \label{equ: Curve Untangling Process}
        \\
        \gammabf(s;0) &= \gammabf_0 (s)
        \label{equ: Initial Curve}
    \end{align}
    where 
    \begin{itemize}
        \item $\gammabf_0 (s)$ is the parameterisation of the initial (tangled) curve (prescribed at $t=0$)
        \item $\mathcal{E}_{\beta}^{\alpha}$ is the tangent-point energy (See (\ref{equ: Tangent-Point Energy}))
        \item $\mathcal{C}$ is additional constraint energy to control behaviour of curve untangling process.
    \end{itemize}
\end{definition}



\subsection{Discretisation for Numerical Computation}
Solving (\ref{equ: Curve Untangling Process}), (\ref{equ: Initial Curve}) analytically is challenging.
Rather, we aim to acquire a numerical solution.
Assume for simplicity that the curve of interest is simple closed.\footnote{
    We may assume that the function definition of $\gammabf:M \rightarrow \mathbb{R}^3$ extends to $\gammabf: \mathbb{R} \rightarrow \mathbb{R}^3$ by periodicity.
}

\subsubsection{Discretisation of Curve}
We start by discretising the initial curve $\gammabf_0$ by taking $N$ points on a curve as shown in Figure \ref{fig: Discretization of Curve}.
Represent the initially discretised curve in the tensor\footnote{``Tensors are multidimensional arrays of numerical values and therefore generalize matrices to multiple dimensions.''\cite{https://doi.org/10.48550/arxiv.1711.10781} Note that we use the definition from machine learning,
rather than mathematics, merely as a convenient data type to capture a discretised curve.}
 form: $\Gammabf^0 = \left( \xbf_0^0, \xbf_1^0, \cdots, \xbf_{N-1}^0 \right)$,
and the discretised curve at subsequent time step $k \in \mathbb{N} \cup \left\{ 0 \right\}$ as $\Gammabf^k = \left( \xbf_0^k, \xbf_1^k, \cdots, \xbf_{N-1}^k \right)$.
Since we restrict our attention to a simple closed curve, it is convenient to extend the indexing rule by:
\begin{equation}
    \xbf_{i}^k = \xbf_{r\left( i,N \right)}^k \hspace{1cm} \text{where } r\left( i,N \right) = \text{(remainder of } i \div N \text{)}
\end{equation}
so that $\xbf_N^k = \xbf_{0}^k$, $\xbf_{N+1}^k = \xbf_{1}^k$, etc.

Define (right) operator $[\cdot]:\mathbb{R}^{3 \times N} \rightarrow \mathbb{R}$ such that for tensor $T \in \mathbb{R}^{3 \times N}$,
%$\Gammabf^k: \mathbb{Z} \rightarrow \left\{ \xbf_0^k, \xbf_1^k, \cdots, \xbf_{N-1}^k \right\}$ as:
\begin{equation*}
    T[i] \coloneqq T \ebf_{r\left( i,N \right)}
\end{equation*}
where $\ebf_i$ is the canonical vector with 1 as its only nontrivial $i$\textsuperscript{th} coordinate.
With this operator, one could write
\begin{equation}
    \Gammabf^k [i] = \xbf_{r\left( i, N \right)}^k = \xbf_i^k
\end{equation}
anaogous to $\gammabf = \gammabf(s)$ being a parameterised curve, which is a vector-valued function.

Finally, denote by $e_{i}^k$ for the (undirected) edge with vertex pair $\left( \xbf_i^k, \xbf_{i+1}^k \right)$.

\begin{remark}
    $\Gammabf^k$ is a tensor in $\mathbb{R}^{3 \times N}$.
\end{remark}

\subsubsection{Discretisation of Tangent-Point Energy}
In order to acquire numerical solution, one must also be able to numerically compute the tangent-point energy.
For this, we pose the energy quadrature $E_{\beta}^{\alpha} \left( \Gammabf \right)$ of the following form:
\begin{equation}
    \mathcal{E}_{\beta}^{\alpha} \left( \gammabf \left( \cdot, t \right) \right) \coloneqq
    \iint_{M^2} k_{\beta}^{\alpha} (\gammabf_x, \gammabf_y) \intd \gamma_x \intd \gamma_y
    \approx
    E_{\beta}^{\alpha} \left( \Gammabf^k \right) \coloneqq
    \sum_{i, j \in \left\{ 0, \cdots, N-1 \right\}} K_{\beta}^{\alpha} \left( i, j \right) \norm{e_i^k} \, \norm{e_j^k}
    \label{equ: Energy Approximation Form}
\end{equation}
where
$K_{\beta}^{\alpha}$ is an approximation of tangent-point kernel $k_{\beta}^{\alpha}$ (which is specified at (\ref{equ: Kernel 4-point})), 
and $\norm{e_i^k}$ is the length of edge $e_i^k$, that is, $\norm{e_i^k} = \norm{\xbf_i^{k+1} - \xbf_i^k}$.
Note that $\Gammabf^k$ is a polygonal curve, for which tangent-point energy (\ref{equ: Tangent-Point Energy}) is not well-defined due to locally non-integrable contributions from vertices:

\begin{figure}[tbp]
    \centering
    \includegraphics[width=0.8\textwidth]{sections/discretizationImgs/discretization}
    \caption{Discretisation of a simple closed curve by sampling the points along the curve.}
    \label{fig: Discretization of Curve}
\end{figure}

One way to resolve the issue is to ``ignore'' the adjacent edge contribution\cite{YSC2021} in the energy quadrature $E_{\beta}^{\alpha}$ as shown in Figure \ref{fig: Energy discretization by ignoring adjacent edges}.
The justification is that as we take a finer mesh ($N$ sufficiently large), the product of edge lengths ($\norm{e_i}\norm{e_j}$) should tend to zero sufficiently fast, resulting in approximation of the energy for the smooth curve, which did not have vertices resulting in local non-integrability in the first place.
\begin{figure}[tbp]
    \centering
    \begin{subfigure}[b]{0.75\textwidth}
        \centering
        \includegraphics[width=\textwidth]{sections/unknottingCurveImgs/energyDiscretization1}
        \caption{For a chosen edge $e_i^k$, ignore the two adjacent edges $e_{i-1}^k$, $e_{i+1}^k$.
            In the limit as $N \rightarrow 0$, because the edge lengths tend to zero,
            the discrepancy between the quadrature and the analytical value of the energy is expected to tend to zero.
                }
        \label{fig: Energy discretization by ignoring adjacent edges}
    \end{subfigure}
    \par\bigskip
    \begin{subfigure}[b]{0.75\textwidth}
        \centering
        \includegraphics[width=\textwidth]{sections/unknottingCurveImgs/energyDiscretization2}
        %\caption{Tangent-point kernel is approximated by 4-point quadrature $K_\beta^\alpha (i, j)$ defined as (\ref{equ: Kernel 4-point}).}
        \caption{Tangent-point kernel is approximated by 4-point quadrature defined as (\ref{equ: Kernel 4-point}).}
        \label{fig: Tangent-point kernel approximation}
    \end{subfigure}
    \caption{Quadrature for approximation of tangent-point energy.}
\end{figure}

It still remains to sensibly approximate the kernel $k_{\beta}^{\alpha} (\gammabf_x, \gammabf_y) \approx K_{\beta}^{\alpha} (i, j)$.
One sensible approximation is to use the following 4-point quadrature\cite{YSC2021}.
\begin{multline}
    K_{\beta}^{\alpha} (i, j) \coloneqq \frac{1}{4} \biggl( k_{\beta}^{\alpha} \left( \xbf_i^k, \xbf_j^k, \Tbf_i^k \right)
        + k_{\beta}^{\alpha} \left( \xbf_i^k, \xbf_{j+1}^k, \Tbf_i^k \right) \\
        + k_{\beta}^{\alpha} \left( \xbf_{i+1}^k, \xbf_j^k, \Tbf_i^k \right)
        + k_{\beta}^{\alpha} \left( \xbf_{i+1}^k, \xbf_{j+1}^k, \Tbf_i^k \right)
    \biggr)
    \label{equ: Kernel 4-point}
\end{multline}
where $\Tbf_{i}^k \coloneqq \frac{\xbf_{i+1}^k - \xbf_i^k}{\norm{\xbf_{i+1}^k - \xbf_i^k}}$ approximates the tangent vector to the curve at $\gammabf_x$.
(See Figure \ref{fig: Tangent-point kernel approximation}.)

Putting (\ref{equ: Energy Approximation Form}) and (\ref{equ: Kernel 4-point}) together,
one can write the \textbf{tangent-point energy quadrature} as:
\begin{equation}
    E_{\beta}^{\alpha} \left( \Gammabf^k \right) \coloneqq \sum_{\substack{i, j \in \left\{ 0, \cdots, N-1 \right\} \\ r \left( i-j,N \right) > 1}} K_{\beta}^{\alpha} (i,j) \norm{e_i^k} \, \norm{e_j^k}
    \label{equ: Tangent-Point Energy Quadrature}
\end{equation}
where $r\left( i-j,N \right)$ is the geodesic distance between $i$ and $j$ in modulo $N$,
characterising the avoidance of adjacent edges on simple closed polygonal curve.



\subsubsection{Finite Difference Scheme of Curve Untangling Process}
Based on (\ref{equ: Curve Untangling Process}), one writes the following finite difference scheme:
\begin{equation}
    \mathcal{D}_{t} \Gammabf^{k} = -\Grad_{X} E_{\beta}^{\alpha} (\Gammabf^k) - \Grad_{X} C \left( \Gammabf^k \right) \hspace{1cm} \text{for } k=0,1,\cdots
    \label{equ: Finite Difference Scheme for Curve Untangling Process}
\end{equation}
where $\mathcal{D}_t$ is the finite difference operator over time,
$\Grad_{X}$ is discrete equivalent\footnote{Worth noting that $\Grad_X:\mathbb{T} \rightarrow \mathbb{T}$ where $\mathbb{T}$ is a set of tensors of certain shape; $\Grad_X$ maps tensors to tensors of the same shape.}  of $\grad_{\mathcal{X}}$ on discrete inner product space $X$ (which may be omitted in the notation),
$E_{\beta}^{\alpha}$ is the tangent-point energy quadrature defined as (\ref{equ: Tangent-Point Energy Quadrature}),
and $C$ is the discretised version of the constraint energy $\mathcal{C}$.
Often times, however, it is easier to construct finite difference scheme directly from (\ref{equ: Curve Untangling Process}) after some simplification,
rather than attempting to use (\ref{equ: Finite Difference Scheme for Curve Untangling Process}) directly.
Note that because similar operations are done for each point vector, this is a parallelisable task.
For the simplest scheme, one could take the forward difference operator characterised as $\mathcal{D}_t \Gammabf^k[i] \coloneqq \frac{\Gammabf^{k+1}[i] - \Gammabf^{k}[i]}{\Delta T}$.

\subsection{Example: $L^2$ Explicit Euler Scheme}
Now we visit the simplest concrete numerical scheme for curve untangling process.
Assume for now scale-invariance by taking parameters $\alpha$ and $\beta$ to satisfy $\beta = \alpha+2$ (See lemma \ref{lemma: Scale-Invariance}).

In $L^2$, $\grad_{L^2} \mathcal{E}_{\beta}^{\alpha} \left( \gammabf \right)$ is simply the ``first-order perturbation'' as in (\ref{equ: L2 Gradient Explicit Form}).
Discrete equivalent is the $\ell^2$ space,
where one may justify this by noting that first-order perturbation in a curve is analogous to perturbing each of the point on its discretisation.
 %$\Grad_{\ell^2} = \nabla_{\Gammabf}$.
Taking $\mathcal{D}_t$ from (\ref{equ: Finite Difference Scheme for Curve Untangling Process}) to be forward difference operator,
\begin{equation}
    \frac{\Gammabf^{k+1} - \Gammabf^k}{\Delta T} = - \Grad_{\ell^2} E_{\beta}^{\alpha} \left( \Gammabf^k \right)
    \label{equ: L2 Explicit Euler}
\end{equation}
where one could explicitly write $\Grad_{\ell^2} E_{\beta}^\alpha \left( \Gammabf^k \right)$ as
\begin{equation}
    \Grad_{\ell^2} E_{\beta}^\alpha \left( \Gammabf^k \right)
    =
    \nabla_{\Gammabf^k} E_{\beta}^\alpha \left( \Gammabf^k \right)
    =
    \begin{pmatrix}
        \frac{\partial}{\partial x_{1,1}} & \frac{\partial}{\partial x_{1,2}} & \cdots & \frac{\partial}{\partial x_{1,N-1}} \\
        \frac{\partial}{\partial x_{2,1}} & \frac{\partial}{\partial x_{2,2}} & \cdots & \frac{\partial}{\partial x_{1,N-1}} \\
        \frac{\partial}{\partial x_{3,1}} & \frac{\partial}{\partial x_{3,2}} & \cdots & \frac{\partial}{\partial x_{3,N-1}}
    \end{pmatrix}
    E_{\beta}^{\alpha} \left( \Gammabf^k \right)
    \label{equ: Gradient in l2}
\end{equation}
where $x_{j,i}$ refers to the $\left( j,i \right)$ coordinate variable for $3 \times N$ tensor.
Note that $\Grad_{\ell^2} E_{\beta}^\alpha \left( \Gammabf^k \right) \in \mathbb{R}^{3 \times N}$ is also a tensor of the same shape as $\Gammabf^k$,
so naturally, most arithmetics\footnote{Such as addition, subtraction, and scalar multiplication.} needed for (\ref{equ: L2 Explicit Euler}) is well-defined.
\begin{remark}
    $\Grad_{\ell^2}$ is in fact a linear operator with respect to its input, energy.
\end{remark}

One could use this exact form of $\Grad_{\ell^2} E_{\beta}^{\alpha} \left( \Gammabf^k \right)$ as given in appendix \ref{sct: Exact Gradient}, but it could be considered cumbersome (even though there are benefits to implementing this as stated in section \ref{sct: L2 Complexity}).
One could alternatively approximate $\Grad_{\ell^2} E_{\beta}^{\alpha} \left( \Gammabf^k \right)$ by central difference scheme, for example:
for $i=0,1,\cdots,N-1$ and $j = 1,2,3$,
\begin{equation}
    \ebf_j \cdot \left( \Grad_{\ell^2} E_{\beta}^{\alpha} \left( \Gammabf^k \right) [i] \right)
    =
    \frac{\partial E_{\beta}^{\alpha} (\Gammabf^k)}{\partial x_{j,i}}
    \approx
    \frac{1}{2 \Delta X} \left( 
        \bar{E}_{\beta}^{\alpha} (i)
        -
        \underaccent{\bar} E_{\beta}^{\alpha} (i)
    \right)
\end{equation}
where
\begin{align*}
    \bar{E}_{\beta}^{\alpha} (i) &\coloneqq E_{\beta}^{\alpha} \left( \left( \xbf_0, \cdots, \xbf_{i-1}, \xbf_i + \Delta X \ebf_{j}, \xbf_{i+1}, \cdots, \xbf_{N-1}\right) \right) \\
    \underaccent{\bar}{E}_{\beta}^{\alpha} (i) &\coloneqq E_{\beta}^{\alpha} \left( \left( \xbf_0, \cdots, \xbf_{i-1}, \xbf_i - \Delta X \ebf_{j}, \xbf_{i+1}, \cdots, \xbf_{N-1}\right) \right)
\end{align*}
Here $\ebf_j$ is the canonical vector of which has entry $1$ at $j$\textsuperscript{th} component.

%We will start with a demonstration with Buck-Orloff energy ($\alpha=2$, $\beta=4$), as one need not take into account of constraint energy $\mathcal{C}$ due to its scale invariance (lemma \ref{lemma: Scale-Invariance})\footnote{One could construct a similar working example for other scale-invariant tangent-point energies.}.

\begin{remark}
Notice that (\ref{equ: L2 Explicit Euler}) can be interpreted as SDM over all the coordinates on the curve!
This is consistent with the motivation of gradient flow equation given in section \ref{sct: Motivation of Gradient Flow Equation}.
\end{remark}

We now attempt the finite difference scheme with a \textit{scale-variant energy}, that is, when $\beta > \alpha + 2$. 
If one attempts the same finite difference scheme as (\ref{equ: L2 Explicit Euler})
one may observe that the curve ``grows'' in size, indefinitely.
This is due to the fact that with $\beta > \alpha + 2$,
by the virtue of lemma \ref{lemma: Scale-Invariance},
energy scales inversely proportional to its scale factor.
(See Figure \ref{fig: Scale Variant})
\begin{figure}[tbp]
    \centering
    \includegraphics{sections/unknottingCurveImgs/scaleVariant}
    \caption{The height represents $\mathcal{E}_{4.5}^{2}$ for circles of different radius. Note that the energy decreases trivially by taking a larger circle.}
    \label{fig: Scale Variant}
\end{figure}
\subsubsection{Constraint Energy}
From lemma \ref{lemma: Scale-Invariance}, in the case that $\beta > \alpha + 2$,
one could trivially minimise tangent-point energy $\mathcal{E}_{\beta}^{\alpha}$ of the curve 
(and by the same logic, $E_{\beta}^{\alpha}$ of the discretised curve, albeit due to increases in edge lengths, would cease to be a ``valid'' quadrature)
by scaling the curve to infinity.
In order to avoid this phenomenon, or to change the behaviour of the flow, one may add additional energy which penalises unwanted behaviours.

Here are some examples of constraint one could take. (Since we are interested in numerical schemes, constraints are expressed in terms of $C$ rather than $\mathcal{C}$.)
%\begin{itemize}
%    \item Taking $\mathcal{C} \coloneqq \lambda \int_{M} \norm{\gammabf}^{p} \intd \gamma$ adds the motivation for the curve not to stray away from the origin.
%    \item Taking $\mathcal{C} \coloneqq \lambda \left( \int_M \intd \gamma - L \right)^p$ adds the motivation for the curve to stay close to some constant arc-length $L$.
%\end{itemize}
\begin{itemize}
    \item Taking $C \left( \Gammabf^k \right) \coloneqq \lambda \sum_{i=0}^{N-1} \norm{\Gammabf^k [i]}^p$ adds the motivation for the curve not to stray away from the origin.
    \item Taking $C \left( \Gammabf^k \right) \coloneqq \lambda \left(\sum_{i=0}^{N-1} \norm{e_i} - L\right)$ adds the motivation for the curve to stay close to some constant arc-length $L$.
\end{itemize}
%For both cases,
$\lambda > 0$ is the strength of the overall constraint, and
$p > 0$ is the order of local strength, where higher $p$ would lead to harsher ``thresholding''.

\subsubsection{Time Complexity}
\label{sct: L2 Complexity}
Since we expect to take $N$ to be very large, it is reasonable to consider the computational work.

If we \textit{use a difference scheme} to approximate $\Grad_{\ell^2} E_{\beta}^\alpha \left( \Gammabf^k \right)$,
for every time step, one needs to be able to evaluate $E_{\beta}^\alpha$ for a given $\Gammabf^k \in \mathbb{R}^{3\times N}$.
From (\ref{equ: Tangent-Point Energy Quadrature}),
it takes $O \left( N^2 \right)$ evaluations of the kernel.
Since the approximation for $\Grad_{L^2} E_\beta^\alpha$ requires evaluation of the energy for each point perturbed, it takes $O \left( N^3 \right)$ evaluations of the kernel for each time step. 

On the other hand, this is where doing the hard work of implementing \textit{exact gradient computation} is beneficial,
For exact gradient implementation as in appendix \ref{sct: Exact Gradient},
it takes $O \left( N \right)$ evaluations\footnote{$4 \left( N-3 \right)$ to be exact.} of the derivative of the kernel.
This implies the overall computation needed for a single step is $O \left( N^2 \right)$.

\subsection{Example: $H^{-1}$ Explicit Euler Scheme}
By (\ref{equ: H-1 Gradient}), we may write $\grad_{H^{-1}} \mathcal{E}_{\beta}^{\alpha} \left( \gammabf \right) = - \laplacian \left( \grad_{L^2} \mathcal{E}_{\beta}^{\alpha} \left( \gammabf \right)\right)$,
where we do not necessarily assume scale-invariance any more.
For the discrete equivalent, take
\begin{equation}
    \Grad E_{\beta}^{\alpha} = - \laplacian_{\Gammabf^k} \left( \Grad_{l^2} E_{\beta}^{\alpha} \left( \Gammabf^k \right) \right) = - \laplacian_{\Gammabf^k} \left( \nabla_{\Gammabf^k} E_{\beta}^{\alpha} \left( \Gammabf^k \right) \right)
\end{equation}

%Taking $\mathcal{D}_t$ from (\ref{equ: Finite Difference Scheme for Curve Untangling Process}) to be forward difference operator,

%\subsection{Example: $H^1$ Explicit Euler Scheme}
%By (\ref{equ: H1 Gradient}), we may write $\grad_{H^1} \mathcal{E}_{\beta}^{\alpha} \left( \gammabf \right) = - \laplacian^{-1} \left( \grad_{L^2} \mathcal{E}_{\beta}^{\alpha} \left( \gammabf \right)\right)$.
%The gradient flow equation with constraint (\ref{equ: Curve Untangling Process}) can be written as,
%\begin{equation}
%    \laplacian \frac{\partial \gammabf}{\partial t} = \grad_{L^2} \mathcal{E}_{\beta}^{\alpha} \left( \gammabf \right) - \laplacian \grad_{L^2} \mathcal{C} \left( \gammabf \right)
%\end{equation}
%The finite difference scheme can be written by approximating $\frac{\partial}{\partial t} \approx \mathcal{D}_t$:
\end{document}
