\documentclass[../dissertation.tex]{subfiles}

\begin{document}
We have defined tangent-point energy quadrature for simple closed curve (any curve homeomorphic to circle) $\gammabf$ at (\ref{equ: Tangent-Point Energy Quadrature}).

For curves in different homeomorphism class from $S^1$,
one may make minor changes to (\ref{equ: Tangent-Point Energy Quadrature}) to get a sensible quadrature.

\subsection{Curves Homeomorphic to a Line Segment}
Given a curve $\gammabf$ homeomorphic to a line segment,
one could take the following as its tangent-point energy quadrature for its discretisation $\Gammabf^k$:
\begin{equation}
    E_{\beta}^{\alpha} \left( \Gammabf^k \right) \coloneqq \sum_{\substack{i,j \in \left\{ 1, \cdots, N-2 \right\} \\ r\left( i-j,N \right) > 1}} K_{\beta}^{\alpha} (i,j) \norm{e_i} \, \norm{e_j}
\end{equation}
where the contribution from each end is neglected.

\subsection{Curves Homeomorphic to Bus Network}
For a curve $\gammabf$ that has ``branches'', one needs to take into account of at what points on the discretisation branching happens.
Suppose for all branches stem from a single curve.
One can capture this by a tensor $\Gammabf$ and a vector containing the indices of point vectors of which are the ``head'' of branches.

\end{document}
