\documentclass[../dissertation.tex]{subfiles}

\begin{document}
\begin{figure}[tpb]
    \centering
    \begin{subfigure}[b]{0.4\textwidth}
        \centering
        \includegraphics[width=\textwidth]{sections/otherHomeomorphismImgs/linesegment}
        \caption{A helix is an example of a curve homeomorphic to a line segment.}
        \label{fig: Line Segment Homeomorphism Class}
    \end{subfigure}
    \hspace{1cm}
    \begin{subfigure}[b]{0.4\textwidth}
        \centering
        \includegraphics[width=\textwidth]{sections/otherHomeomorphismImgs/branch}
        \caption{A curve with branches from a main ``stem''.}
        \label{fig: Bus Network}
    \end{subfigure}
    \caption{Curves in different homeomorphism classes.}
\end{figure}
We have defined tangent-point energy quadrature for simple closed curve (any finite curve homeomorphic to unit circle $S^1$) $\gammabf$ at (\ref{equ: Tangent-Point Energy Quadrature}).

For curves in different homeomorphism class from $S^1$,
one may make minor changes to (\ref{equ: Tangent-Point Energy Quadrature}) to get a sensible quadrature.

\subsection{Curves Homeomorphic to a Line Segment}
Given a curve $\gammabf$ homeomorphic to a line segment such as the helix (Figure \ref{fig: Line Segment Homeomorphism Class}),
one could take the following as its tangent-point energy quadrature for its discretisation $\Gammabf^k$:
\begin{equation}
    E_{\beta}^{\alpha} \left( \Gammabf^k \right) \coloneqq \sum_{\substack{i,j \in \left\{ 1, \cdots, N-2 \right\} \\ r\left( i-j,N \right) > 1}} K_{\beta}^{\alpha} (i,j) \norm{e_i} \, \norm{e_j}
\end{equation}
where the contribution from each end is neglected.
Note that unlike a simple closed curve,
one does not generalise the indexing rule to be ``cyclic''.

\subsection{Curves Homeomorphic to ``Bus Network''}
For a curve $\gammabf$ that has $m$ ``branches'', one needs to take into account of at what points on the discretisation branching happens.
Suppose for all branches protrude from a single ``stem''.
On Figure \ref{fig: Bus Network}, the stem is represented by the thick blue curve.

One can capture this curve by a tensor of the following form.
\begin{equation}
    \Gammabf^k = \left( \underbrace{\xbf_{0,0}^k, \cdots, \xbf_{0,n_0-1}^k}_{\text{Stem}} \middle|
    \underbrace{\xbf_{1,0}^k, \cdots, \xbf_{1,n_1-1}^k}_{\text{Branch 1}}
\middle| \cdots \middle|
\underbrace{\xbf_{m,0}^k, \cdots, \xbf_{m,n_m-1}^k}_{\text{Branch } m}\right)
\label{equ: Bus Network Tensor}
\end{equation}
where the ``linkage points'' for branches to the stem are $\xbf_{1,0}^k, \cdots, \xbf_{m,0}^k$,
which can be identified by the fact that $\xbf_{1,0}^k, \cdots, \xbf_{m,0}^k \in \left\{ \xbf_{0,0}^k, \cdots, \xbf_{0, n_0-1}^k \right\}$,
that is, all duplicate points are to be understood as linkage points.
Denote the multiset\footnote{``a set that allows multiplicity of same elements''} of linkage points as $L \left( \Gammabf^k \right)$.
Then one may note the relation $N = \sum_{i=0}^{m} n_{i} - |L\left( \Gammabf^k \right)| = \sum_{i=0}^m n_i - m$.
Note that this is a natural generalisation of discretisation of curves homeomorphic to a line segment.

Now, one may write a tangent-point energy quadrature as:
\begin{equation}
    E_{\beta}^{\alpha} \left( \Gammabf^k \right) \coloneqq 
    \sum_{p, q \in \left\{ 0, 1, \cdots, m \right\}}
    \sum_{\substack{
            i \in \left\{ 1, \cdots, n_p - 2 \right\} \\
            j \in \left\{ 1, \cdots, n_q - 2 \right\} \\
            \sigma \left( e_{p_i}, e_{q_j} \right) = 1
    }}
    K_{\beta}^{\alpha} (p_i,q_j) \norm{e_{p_i}} \, \norm{e_{q_j}}
    \label{equ: Bus Network Quadrature}
\end{equation}
where $p_i$ refer to the index of vector $\xbf_{p,i}^k$ in (\ref{equ: Bus Network Tensor}),
that is,
\begin{align*}
    p_i = \sum_{j=0}^{p-1} n_j + i
\end{align*}
(similarly with $q_j$),
and
\begin{equation*}
    \sigma \left( e_1, e_2 \right) =
    \begin{cases}
        0 & \exists \text{ shared vertex between edges } e_1, e_2 \\
        1 & \text{Otherwise}
    \end{cases}
\end{equation*}

\begin{remark}
    Representation (\ref{equ: Bus Network Tensor}) is not unique; for example, one could take different curve to be the ``stem''.
    The quadrature given by (\ref{equ: Bus Network Quadrature}) is, however, invariant under different representation, hence well-defined.
\end{remark}


\end{document}
