\documentclass[../dissertation.tex]{subfiles}
\begin{document}
Instead of solving a discretised gradient flow equation (\ref{equ: Finite Difference Scheme for Curve Untangling Process}) directly,
a possible approach is to use approximation theory for a simple class of tangled curves;
if one knows beforehand that the initial curve is homeomorphic to a circle or a line segment,
it is possible to interpret and approximate the curve as comprised of simple functions.

\subsubsection{Homeomorphic to a Circle}
For a continuous $2\pi$-periodic 1D function $f:\mathbb{R} \rightarrow \mathbb{R}$
(where one only needs to define $f$ on $\left[ 0,2\pi \right)$),
there exists a Fourier series representation:
\begin{align}
    f(x) &= \frac{a_0}{2} + \sum_{n=1}^{\infty} \left( a_n \cos {(nx)} + b_n \sin {(nx)} \right) \\
    &= \sum_{n=-\infty}^{\infty} c_n e^{inx}
\end{align}
where the two representations are equivalent.
The coefficients are given by
\begin{align}
    a_n &= \frac{1}{\pi} \int_{0}^{2\pi} f(x) \cos {\left( nx \right)} \intd x &\in \mathbb{R}\\
    b_n &= \frac{1}{\pi} \int_{0}^{2\pi} f(x) \sin {\left( nx \right)} \intd x &\in \mathbb{R} \\
    c_n &= \frac{1}{2 \pi} \int_{0}^{2\pi} f(x) e^{-inx} \intd x &\in \mathbb{C}
\end{align}

Now, for a vector-valued function such as a parameterised curve $\gammabf:\mathbb{R} \rightarrow \mathbb{R}^3$, one could write 3D Fourier series,
one for each coordinate:
\begin{align}
    \mathbf{\gammabf} (t) &= \frac{1}{2}
    \begin{pmatrix}
        a_{1,0} \\
        a_{2,0} \\
        a_{3,0}
    \end{pmatrix}
    + \sum_{n=1}^\infty
    \begin{pmatrix}
        a_{1,n} & b_{1,n} \\
        a_{2,n} & b_{2,n} \\
        a_{3,n} & b_{N,n}
    \end{pmatrix}
    \begin{pmatrix}
        \cos {\left( nt \right)} \\
        \sin {\left( nt \right)}
    \end{pmatrix}
    \\
    &= \sum_{n=-\infty}^{\infty}
    \begin{pmatrix}
        c_{1,n} \\
        c_{2,n} \\
        c_{3,n}
    \end{pmatrix}
    e^{-int}
\end{align}
where the coefficients $\left\{ a_{i,n} \right\}, \left\{ b_{i,n} \right\}, \left\{ c_{i,n} \right\}$ are given by
\begin{align}
    a_{i,n} &= \frac{1}{\pi} \int_{0}^{2\pi} f_i (x) \cos {\left( nx \right)} \intd x & \in \mathbb{R}\\
    b_{i,n} &= \frac{1}{\pi} \int_{0}^{2\pi} f_i (x) \sin {\left( nx \right)} \intd x & \in \mathbb{R}\\
    c_{i,n} &= \frac{1}{2\pi} \int_{0}^{2\pi} f_i (x) e^{-inx} \intd x & \in \mathbb{C}
\end{align}
for $i=1, 2, 3$.

Now the idea is to truncate the series to order $J > 0$ terms and collect its coefficients,
that is,
\begin{equation*}
    \gammabf(t) \approx \sum_{n=-J}^{J}
    \begin{pmatrix}
        c_{1,n} \\
        c_{2,n} \\
        c_{3,n}
    \end{pmatrix}
    e^{-int}
\end{equation*}
One may justify the truncation of Fourier series representation by the following theorem:
\begin{theorem}[Fourier Convergence Theorem]
\lipsum[1]
\end{theorem}
Now, using these coefficients, one could construct the approximate discrete curve $\Gammabf_{N,J}$ of resolution $N$.
This time, instead of solving for gradient flow equation,
we use standard methods of minimising a function, such as gradient descent or otherwise \textit{over the coefficients}.
The smoothness of the evolving curve follows since it is represented as a finite sum of infinitely smooth function,
so there is no misbehaviour as in $L^2$ gradient flow.

\end{document}
