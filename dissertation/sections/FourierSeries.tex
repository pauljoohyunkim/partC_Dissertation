\documentclass[../dissertation.tex]{subfiles}
\begin{document}
Instead of solving a discretised gradient flow equation (\ref{equ: Finite Difference Scheme for Curve Untangling Process}) directly,
a possible approach is to use approximation theory for a simple class of tangled curves;
if one knows beforehand that the initial curve is homeomorphic to a circle or a line segment,
it is possible to interpret and approximate the curve as comprised of simple functions.
Denote by $\mathbb{S}$ and $\mathbb{L}$ for bounded curves that are homeomorphic to a circle and a line segment, respectively, as in Appendix \ref{sct: Other Homeomorphism Classes}.

\subsubsection{Curves in $\mathbb{S}$}
For a continuous $2\pi$-periodic 1D function $f:\mathbb{R} \rightarrow \mathbb{R}$
(where one only needs to define $f$ on $\left[ 0,2\pi \right)$),
there exists a Fourier series representation:
\begin{align}
    f(x) &= \frac{a_0}{2} + \sum_{n=1}^{\infty} \left( a_n \cos {(nx)} + b_n \sin {(nx)} \right) \\
    &= \sum_{n=-\infty}^{\infty} c_n e^{inx}
\end{align}
where the two representations are equivalent.
The coefficients are given by
\begin{align}
    a_n &= \frac{1}{\pi} \int_{0}^{2\pi} f(x) \cos {\left( nx \right)} \intd x &\in \mathbb{R}\\
    b_n &= \frac{1}{\pi} \int_{0}^{2\pi} f(x) \sin {\left( nx \right)} \intd x &\in \mathbb{R} \\
    c_n &= \frac{1}{2 \pi} \int_{0}^{2\pi} f(x) e^{-inx} \intd x &\in \mathbb{C}
\end{align}

Now, for a vector-valued function such as a parameterised curve $\gammabf:\mathbb{R} \rightarrow \mathbb{R}^3$, one could write 3D Fourier series,
one for each coordinate:
\begin{align}
    \mathbf{\gammabf} (t) &= \frac{1}{2}
    \begin{pmatrix}
        a_{1,0} \\
        a_{2,0} \\
        a_{3,0}
    \end{pmatrix}
    + \sum_{n=1}^\infty
    \begin{pmatrix}
        a_{1,n} & b_{1,n} \\
        a_{2,n} & b_{2,n} \\
        a_{3,n} & b_{N,n}
    \end{pmatrix}
    \begin{pmatrix}
        \cos {\left( nt \right)} \\
        \sin {\left( nt \right)}
    \end{pmatrix}
    \\
    &= \sum_{n=-\infty}^{\infty}
    \begin{pmatrix}
        c_{1,n} \\
        c_{2,n} \\
        c_{3,n}
    \end{pmatrix}
    e^{int}
\end{align}
where the coefficients $\left\{ a_{i,n} \right\}, \left\{ b_{i,n} \right\}, \left\{ c_{i,n} \right\}$ are given by
\begin{align}
    a_{i,n} &= \frac{1}{\pi} \int_{0}^{2\pi} \gamma_i (t) \cos {\left( nt \right)} \intd t & \in \mathbb{R}\\
    b_{i,n} &= \frac{1}{\pi} \int_{0}^{2\pi} \gamma_i (t) \sin {\left( nt \right)} \intd t & \in \mathbb{R}\\
    c_{i,n} &= \frac{1}{2\pi} \int_{0}^{2\pi} \gamma_i (t) e^{-int} \intd t & \in \mathbb{C}
\end{align}
for $i=1, 2, 3$ where $\gamma_i$ represents $i$\textsuperscript{th} coordinate of $\gammabf$.

Now the idea is to truncate the series to order $J > 0$ terms and collect its coefficients,
that is,
\begin{equation*}
    \gammabf(t) \approx \sum_{n=-J}^{J}
    \begin{pmatrix}
        c_{1,n} \\
        c_{2,n} \\
        c_{3,n}
    \end{pmatrix}
    e^{-int}
\end{equation*}
One may justify the truncation of Fourier series representation by the following theorem:
\begin{theorem}[Fourier Convergence Theorem]
\lipsum[1]
\end{theorem}
Now, using these coefficients, one could construct the approximate discrete curve $\Gammabf_{N,J}$ of resolution $N \gg J$.
Take the dependent variables for the tangent-point energy to be the coefficients on Fourier series representation:
\begin{equation}
    \mathcal{E}_{\beta}^{\alpha} \left( \gammabf \right)
    \approx
    \mathcal{E}_{\beta}^{\alpha} \left( \gammabf_{J} \right)
    =
    \mathcal{E}_{\beta}^{\alpha} \underbrace{\left( \cbf_{-J}, \cbf_{-J+1}, \cdots, \cbf_{J} \right)}_{3\left( 2J+1 \right) \text{ variables}}
    \approx
    E_{\beta}^{\alpha} \left( \Gammabf_{N,J} \right)
\end{equation}
This time, instead of solving for gradient flow equation,
we use standard methods of minimising a function, such as gradient descent or otherwise \textit{over the $3(2J+1)$ coefficients\footnote{Without loss of generality, one could take $c_{i,0} = 0$ for $i=1,2,3$ as it only determines constant shift of the entire curve, so one actually only needs to consider $6J$ coefficients.}} $\cbf_{-J}, \cbf_{-J+1}, \cdots, \cbf_{J}$.
It is practical to take $N$ to be larger than the resolution for the discrete gradient flow,
as we have less variables to minimise over.
The smoothness of the evolving curve follows since it is represented as a finite sum of infinitely smooth function,
so there is no misbehaviour from sharp points as in $L^2$ gradient flow.

\subsubsection{Curves in $\mathbb{L}$}
For $\gammabf$ that is not closed, but rather homeomorphic to a line segment,
it is more natural to use Chebyshev approximation\cite{Trefethen_2020}.
For a curve $\gammabf:\left[ -1, 1 \right] \rightarrow \mathbb{R}^3$ such that $\gammabf \in \mathbb{L}$,
\begin{equation}
    \gammabf (t) =
    \begin{pmatrix}
        a_{1,0} \\
        a_{2,0} \\
        a_{3,0}
    \end{pmatrix}
    +
    \sum_{n=1}^{\infty}
    \begin{pmatrix}
        a_{1,n} \\
        a_{2,n} \\
        a_{3,n}
    \end{pmatrix}
    T_n (t)
\end{equation}
where $T_n (t)$ is the $n$\textsuperscript{th} Chebyshev polynomial.
The coefficients $\left\{ a_{i,n} \right\}$ are given by:
\begin{equation}
    a_{i,n} = \frac{2}{\pi} \int_{-1}^{1} \frac{\gamma_i(t) T_n(t)}{\sqrt{1-t^2}} \intd t
\end{equation}
for $i = 1, 2, 3$.

We again approximate $\gammabf$ by truncating the Chebyshev series up to order $J$ term,
and interpret the tangent-point energy to dependent on its coefficients.
\begin{equation}
    \mathcal{E}_{\beta}^{\alpha} \left( \gammabf \right)
    \approx
    \mathcal{E}_{\beta}^{\alpha} \left( \gammabf_{J} \right)
    =
    \mathcal{E}_{\beta}^{\alpha} \underbrace{\left( \abf_{0}, \abf_{1}, \cdots, \abf_{J} \right)}_{3\left( J+1 \right) \text{ variables}}
    \approx
    E_{\beta}^{\alpha} \left( \Gammabf_{N,J} \right)
\end{equation}
This time, the tangent-point quadrature is taken as the one in Appendix \ref{sct: Curves in L}.

\subsubsection{Sample Points}
While approximation via series reduces the number of dimensions to reduce the energy over,
in order to compute the derivative (against coefficients) of the tangent-point energy,
one still needs to use quadrature, meaning, one needs to take $N$ sample points.

Because we have captured the function that describes the curve rather than points,
there is more flexibility to either sample/resample points as needed.

There are, again, natural choice of points.

\end{document}
