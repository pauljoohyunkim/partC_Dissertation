\documentclass[../dissertation.tex]{subfiles}

\begin{document}
(Adapted based on definitions given in lecture note by S\"uli\cite{nspde}.
For our purposes, we focus our attention to interval $I \subset \mathbb{R}$.)
Here are some of the notable inner product spaces.
\subsection{$L^2$ Space}
\begin{definition}[$L^2$ Space]
    For interval $I \subset \mathbb{R}$,
    $L^2$ space over is defined as
    \begin{equation}
        L^2 = \left\{ f:I \rightarrow \mathbb{R} \middle| f \text{ measurable}, \left( \int_{I} |f|^2 \intd x \right)^{1/2} < \infty  \right\}
    \end{equation}

    $L^2$ inner product is defined as
    \begin{equation}
        \forall f,g \in L^2 \; \; \inner{f, g}_{L^2} = \int_I f g \intd x
    \end{equation}
\end{definition}

\subsection{$H^k$ Space}
To define Sobolev (inner product) spaces (denoted $H^k$ where $k \in \mathbb{N} \cup \left\{ 0 \right\}$)
one must define weak derivative operator $D$:
\begin{definition}[Weak Derivative]
    For $u$ locally integrable on $I$, if there exists $w$ such that for all infinitely smooth $v:I \rightarrow \mathbb{R}$ with compact support,
    \begin{equation}
        \int_{I} w v \intd x = (-1)^{\alpha} \int_{I} u \frac{\intd^{\alpha} v}{\intd x^{\alpha}} \intd x
    \end{equation}
    then $w$ is said to be \textbf{weak derivative} of order $\alpha$ of $u$, and one writes $D^{\alpha}u = w$.
\end{definition}

Weak derivative extends the definition of conventional derivative, and is equivalent to the conventional derivative for smooth functions.
With weak derivatives introduced, one may now define Sobolev inner product spaces.
\begin{definition}[$H^k$ Space]
    Sobolev inner product space of order $k \in \mathbb{N} \cup \left\{ 0 \right\}$ (denoted $H^k$) is defined as
    \begin{equation}
        H^k = \left\{ f \in L^2 \left.\middle|\right. D^{\alpha} f \in L^2, \alpha \leq k \right\}
    \end{equation}

    $H^k$ inner product is defined as:
    \begin{equation}
        \forall f,g \in H^{k} \; \; \inner{f, g}_{H^2} = \sum_{\alpha \leq k} \inner{D^\alpha f, D^\alpha g}_{L^2}
    \end{equation}
\end{definition}

\begin{remark}
    Note that $H^0 = L^2$ by definition.
    One could say that Sobolev inner product spaces extend $L^2$ space.
    It also turns out that $H^k$ are Hilbert spaces.
\end{remark}


\end{document}
