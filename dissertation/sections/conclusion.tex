\documentclass[../dissertation.tex]{subfiles}
\begin{document}
Because of the nature of discretisation, there is no absolute guarantee that a curve would not intersect;
this implies that the implementation of the curve untangling process must be carefully designed.
Throughout experiments, common points of failure were:
\begin{itemize}
    \item Some of the edge lengths of the discretised curve became sufficiently large that the quadrature ceased to become valid.
    \item The curves ceased to be smooth (as in Figure \ref{fig: L2 Curve Unknotting} partially).
    \item The curves ``collapsing'' to nullify the kernel, usually after a long time since reaching a stationary state.
    \item Self-intersection in some cases.
\end{itemize}

The simplest idea to resolve some of the arising issues is to take a combination of appropriate constraints as in section \ref{sct: Constraint Energy} to steer the curve away from those issues.
For example, the first point of failure could be fixed by taking the constraint of prescribing edge lengths to be the initial edge lengths. 
Of course, the other points of failure are innate to the approach itself,
and one must tweak the scheme to resolve them.
The nonsmoothness issue can be resolved by using a higher order Sobolev space,
Nullification of the kernel can be resolved by stopping the evolution afer achieving the energy value within a specified tolerance from the energy of the stationary state;
for example, if one attempts evolution of curve which the unknot is a circle, with Buck-Orloff tangent-point energy $\mathcal{E}_{4}^{2}$, we stop the evolution after the energy of the curve is within $\epsilon > 0$ from $\pi^2$,
the Buck-Orloff tangent-point energy of a circle.
For the issue of self-intersection,
it turns out that in practice, taking higher resolution and smaller time step ends up resolving it.

Of course, there are other possible things one could do instead to ameliorate the result of evolution.

\subsection{Possible Variations to Curve Untangling Process}
The most obvious way to improve the evolution is to take smaller time step and higher resolution of the curve.
However, reducing the time step means the evolution itself will be slower, which is undesirable.

One modification to discrete curve untangling process is to \textbf{vary the time step};
instead of using a fixed time step $\Delta T > 0$, one could try modified backtracking Armijo line search.
Take initial time step $\Delta T^{0}$ to be such that the evolution would result in a collision.



\subsection{Discussion of Fractional Sobolev Space}
While this dissertation focused on $L^2$ and $H^1$ gradient flow for the untangling process,
for given $\alpha$ and $\beta$ values for the tangent-point energy,
there is an optimal choice of space for one to implement the curve untangling process,
which makes the evolution by (\ref{equ: Finite Difference Scheme for Curve Untangling Process}) the fastest
and its computation trivial.

\end{document}
