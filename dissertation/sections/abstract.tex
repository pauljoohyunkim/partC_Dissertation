\documentclass[../dissertation.tex]{subfiles}
\begin{document}
\thispagestyle{plain}
\begin{center}
    \Large
    \textbf{\titlename}
        
    %\vspace{0.4cm}
    %\large
    %Thesis Subtitle
        
    %\vspace{0.4cm}
    %\textbf{Author Name}
       
    \vspace{0.9cm}
    \textbf{Abstract}
\end{center}
Curves are one of the fundamental objects in geometry and engineering,
yet most analysis of curves often disregard their physical characteristics such as their spatial volume or uncrossability.
One common situation that such physical characteristics become significant is when one attempts to untangle a knot.
An approach to achieve this is to assign an ``energy'' to a curve such that
this energy would increase when two points on ``different sides'' of a curve are closer,
then one continuously deforms the curve to reduce this energy, 
the expectation being that the curve that achieves minimal energy must be the untangled knot.
This dissertation explores numerical methods of achieving this.

\vspace{0.2cm}
\hrule

\end{document}
