\documentclass[../dissertation.tex]{subfiles}

\begin{document}

\subsection{Steepest Descent to Gradient Flow Equation}
For minimising a differentiable function $f:E \subset \mathbb{R}^n \rightarrow \mathbb{R}$, there is a well-known method known as \textbf{steepest descent method} (SDM)\cite{doi:10.1137/1.9781611974997.ch8}.
\begin{equation}
    \xbf^{k+1} = \xbf^{k} - \alpha_k \nabla f \left( \xbf^{k} \right)
    \label{equ: SDM}
\end{equation}
Starting from the initial input point $\xbf^{0}$, at each iteration, input points $\left\{ \xbf^{k} \right\}$ move in the direction of ``steepest'' decrease,
with specified step size $\alpha_k > 0$,
reducing the value at evaluation of $f$.
\begin{figure}[tbp]
    \centering
    \includegraphics[scale=0.6]{sections/preambleImgs/sdm}
    \caption{SDM applied to $f(x,y) = -3 \cos x + \cos^2 y$ at different initial points.}
    \label{fig: SDM}
\end{figure}
Note that in general, this method is not guaranteed to find the minimiser.
On the other hand, convergence is guaranteed under certain assumptions, for example, convexity and $L$-smoothness with a certain choice of step size $\alpha_k$.

Analogously, differential equation known describing the reduction process of a functional $F:\mathcal{F} \rightarrow \mathbb{R}$ (where $\mathcal{F}$ is some set of functions) can be motivated.
Starting from (\ref{equ: SDM}), replacing $\xbf^{k}$ to $f_k$ and $\nabla f\left( \xbf^k \right)$ to $\grad_X F\left( f_k \right)$
\begin{equation}
    f_{k+1} = f_k - \alpha_k \grad_X F \left( f_k \right)
    \label{equ: Gradient Flow Motivation}
\end{equation}
Now think of $f_k$ as ``snapshots'' at certain time $t = t_k$.
Without loss of generality, let $\alpha_k \equiv 1$.\footnote{
This is justified by taking a differrent time scale; essentially nondimensionalisation.}
Dividing (\ref{equ: Gradient Flow Motivation}) by time step $\Delta t \coloneqq t_{k+1} - t_k$,
and taking the limit as $\Delta t \rightarrow 0$,
we acquire the \textbf{gradient flow equation}\cite{YSC2021}.
\begin{equation}
    \frac{\partial F}{\partial t} = - \grad_X F
\end{equation}
where index $k$ transforms to ``time'' variable $t$.

Note that $\grad$ of a functional is not defined yet. This depends on the function space $X$ of interest.

\end{document}
