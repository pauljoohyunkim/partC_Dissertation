\documentclass[../dissertation.tex]{subfiles}

\begin{document}
Gradient of a function $f:\mathbb{R}^n \rightarrow \mathbb{R}$ characterises the first-order variation in a certain direction,
that is,
\begin{equation}
    \forall \xbf, \ybf \in \mathbb{R}^n \; \; \;
    \nabla f \left( \xbf \right) \cdot \ybf = \lim_{\epsilon \rightarrow 0} \frac{1}{\epsilon} \left( f\left( \xbf + \epsilon \ybf \right) - f\left( \xbf \right)  \right) 
\end{equation}
While this is not a conventional definition of function gradient, it is an appropriate definition.

One could analogously construct the definition of gradient of a functional.
\begin{definition}[Gradient of Functional]
    For a functional $F:\mathcal{X} \rightarrow \mathbb{R}$, define functional gradient $\grad_{\mathcal{X}} F (f)$ as:
    \begin{equation}
        \inner{\grad_{\mathcal{X}} F (f), g}_{\mathcal{X}} = \lim_{\epsilon \rightarrow 0} \frac{1}{\epsilon} \left( F\left( f + \epsilon g \right) - F\left( f \right) \right)
    \end{equation}
    where $\inner{\cdot, \cdot}_{\mathcal{X}}$ is the inner product defined over the inner product function space $\mathcal{X}$.
\end{definition}

Common inner product spaces include $L^2$, $H^1$, \ldots

\end{document}
