\documentclass[../dissertation.tex]{subfiles}

\begin{document}
In order to understand the gradient of a functional, it helps to recall the gradient of a function.
Gradient of a function $f:\mathbb{R}^n \rightarrow \mathbb{R}$ characterises the first-order variation in a certain direction,
that is,
\begin{equation}
    \forall \xbf, \ybf \in \mathbb{R}^n \; \; \;
    \nabla f \left( \xbf \right) \cdot \ybf = \frac{\partial}{\partial \epsilon} f\left( \xbf + \epsilon \ybf \right)|_{\epsilon=0}
\end{equation}
While this is not a conventional definition of function gradient, it is an equivalent definition.

One could analogously motivate the definition of gradient of a functional.
\begin{definition}[Gradient of Functional]
    For a functional $F:\mathcal{X} \rightarrow \mathbb{R}$, define functional gradient $\grad_{\mathcal{X}} F (f)$ as:
    \begin{equation}
        \forall f, g \in \mathcal{X} \; \; \inner{\grad_{\mathcal{X}} F (f), g}_{\mathcal{X}} = \left.\frac{\partial}{\partial \epsilon} F \left( f + \epsilon g \right)\right|_{\epsilon = 0}
        \label{equ: Functional Gradient Definition}
    \end{equation}
    where $\inner{\cdot, \cdot}_{\mathcal{X}}$ is the inner product defined over the inner product function space $\mathcal{X}$.
\end{definition}

Common inner product spaces include $L^2$, $H^1$, which are introduced in Appendix \ref{sct: Definitions of Important Inner Product Spaces}.

\begin{remark}
    By assuming time-independence of $f$ in the gradient flow equation (\ref{equ: Gradient Flow Equation}), we get a stationary state equation $\grad_{\mathcal{X}} F(f) = 0$,
    which implies $\frac{\partial}{\partial \epsilon} F \left( f + \epsilon g \right)|_{\epsilon=0} = 0$ for all $g \in \mathcal{X}$.
    This shows that the stationary state of the gradient flow equation is precisely the solution to the Euler-Lagrange equation.
\end{remark}

\subsubsection{Gradient on Integer Sobolev Spaces}
\label{sct: Gradient on Integer Sobolev Spaces}
Assume $\Omega$ is a boundary-free domain.
Given $L^2(\Omega) = H^0(\Omega)$ gradient ($\Omega$ will be omitted when unambiguous), one could express Sobolev gradients of other orders.
For functional $F$, write $h \coloneqq \grad_{L^2} F$ for gradient of $F$ in $L^2$, that is, by definition
\begin{equation}
    \left.\frac{\partial}{\partial \epsilon} F \left( f + \epsilon g \right)\right|_{\epsilon = 0} = \inner{h, g}_{L^2} \eqqcolon \mathcal{P}
    \label{equ: L2 Gradient Explicit Form}
\end{equation}
By lemma \ref{lemma: Shifting Gradient Operator} and lemma \ref{lemma: Shifting Laplacian Operator} on boundary-free domain $\Omega$,\footnote{Integration by parts to ``shift'' operators to other function.}
\begin{align*}
    \mathcal{P} &= \inner{\laplacian \laplacian^{-1} h, g}_{L^2} \\
    &= \inner{-\nabla \laplacian^{-1} h, \nabla g}_{L^2} \\
    &= \inner{-\laplacian^{-1} h, g}_{H^1}
\end{align*}
So one could spot that
\begin{equation}
    \grad_{H^1} F = -\laplacian^{-1} h
    \label{equ: H1 Gradient}
\end{equation}
\begin{remark}
However, there is a subtlety here.
Because inverting a Laplacian requires additional information
(often about the boundary, even though we assume it was a boundary-free domain),
$\laplacian^{-1}$ can be problematic.
One could resolve this by regularising $\laplacian^{-1}$ to $\left(\laplacian + c \idf\right)^{-1}$
where $c\neq 0$ is a constant and $\idf$ is the identity operator.
One justifies this replacement by noting the fact that $H^1$ gradient is only defined up to
addition of null space of the Laplacian\footnote{``Harmonic functions''}, and addition of $c \idf$ can make the operator invertible.
%Note that $\left(\laplacian + c \idf\right)$ is an invertible operator now.
\end{remark}

Similarly, one could acquire gradient in $H^2$ by
\begin{align*}
    \mathcal{P} &= \inner{\laplacian^{-1} h, \laplacian g}_{L^2} \\
    &= \inner{\laplacian \left( \laplacian^{-2} h \right), \laplacian g}_{L^2} \\
    &= \inner{\laplacian^{-2} h, g}_{H^2}
\end{align*}
Hence,
\begin{equation}
    \grad_{H^2} F = \laplacian^{-2} h
    \label{equ: H2 Gradient}
\end{equation}
One could extend by considering dual space\cite{YSC2021} and define gradient in $H^{-1}$,
\begin{align*}
    \mathcal{P} &= \inner{h, \laplacian \laplacian^{-1} g}_{L^2} \\
    &= \inner{-\nabla h, \nabla \left( \laplacian^{-1} g \right)}_{L^2} \\
    &= \inner{-\laplacian h, g}_{H^{-1}}
\end{align*}
So,
\begin{equation}
    \grad_{H^{-1}} F = -\laplacian h
    \label{equ: H-1 Gradient}
\end{equation}
\begin{remark}
    From (\ref{equ: H1 Gradient}), (\ref{equ: H2 Gradient}), and (\ref{equ: H-1 Gradient}), one may deduce that
    choosing the right Sobolev space for a functional may make the gradient flow equation much easier to solve.
\end{remark}

\begin{example}[Dirichlet Energy]
    For example, given functional $\mathcal{E} (f) \coloneqq \int_{\Omega} |\nabla f \left( \xbf \right)|^2 \intd V$ over an open set $\Omega \subset \mathbb{R}^n$,
    one can explicitly deduce $L^2$ gradient from definition (\ref{equ: Functional Gradient Definition}):
    \begin{align*}
        \inner{\grad_{L^2} \mathcal{E} (f), g}_{L^2} &= \left.\frac{\partial}{\partial \epsilon} \mathcal{E} \left( f + \epsilon g \right)\right|_{\epsilon=0} \\
            &= \left.\frac{\partial}{\partial \epsilon} \int_{\Omega} \left( \nabla f + \epsilon \nabla g \right) \cdot \left( \nabla f + \epsilon \nabla g \right) \intd V\right|_{\epsilon=0} \\
                &= \int_{\Omega} \nabla f \cdot \nabla g \intd V \\
                &\stackrel{\text{IBP}}{=} \underbrace{\oint_{\partial \Omega} g \frac{\partial f}{\partial n} \intd S}_{\text{Boundary Term}} - \int_{\Omega} g \laplacian f \intd V \\
                &= \underbrace{\oint_{\partial \Omega} g \frac{\partial f}{\partial n} \intd S}_{\text{Boundary Term}} + \inner{-\laplacian f, g}_{L^2}
            \end{align*}
            Assuming natural boundary condition $\frac{\partial f}{\partial n} = 0$, or restricting our interest to periodic functions such that there is no boundary, we may set the boundary term to zero.
            We can then read off $L^2$ gradient to be $\grad_{L^2} \mathcal{E} (f) = h = -\laplacian f$. 
            The gradient flow equation turns out to be the heat equation:
            \begin{equation}
                \frac{\partial f}{\partial t} = \laplacian f
            \end{equation}
            (Refer to Figure \ref{fig: GF} for the case $\Omega = (-1, 1)$.)
            However, note that one could ``reduce the order'' or the RHS by considering the gradient flow in $H^1$.
            Observe from (\ref{equ: H1 Gradient}) that $\grad_{H^1} \mathcal{E} (f) = \idf_{\Omega}$, where $\idf_{\Omega}$ is the identity function over $\Omega$.
            Solving $H^1$ gradient flow equation
            \begin{equation}
                \frac{\partial f}{\partial t} = -f
    \end{equation}
    is much easier as it boils down to solving an ordinary differential equation,
    which the solution decays (or rather converges) exponentially fast.

    On the other hand, what if one reduces the order too much by taking $H^2$?
    Since $\grad_{H^2} \mathcal{E} (f) = -\laplacian^{-1} f$, the $H^2$ gradient flow equation becomes:
    \begin{equation}
        \frac{\partial f}{\partial t} = \laplacian^{-1} f
    \end{equation}
    This is solved by solving a Poisson's equation $\laplacian g = f$ for $g$,
    then $g = \frac{\partial f}{\partial t}$,
    which is less practical.
\end{example}

\subsection{Derivative Operators for Curves}
For application of functional gradients on curves $\gammabf: M \rightarrow \mathbb{R}^3$,
the following derivative operators replace the traditional derivative operators,
\begin{definition}[First Derivative Operator ``Curve Gradient'']
    Define first derivative operator\cite{YSC2021} $\curvenabla_{\gammabf}$ in a way that for $u \left( \gammabf (\cdot) \right): M \rightarrow \mathbb{R}$,
    \begin{equation}
        \curvenabla_{\gammabf} u \left( \gammabf \left( s \right) \right)
        = \frac{\intd u \left( \gammabf \left( s \right) \right)}{\norm{\intd \gammabf}} \frac{\intd \gammabf}{\norm{\intd \gammabf}}
        = \frac{\intd u \left( \gammabf \left( s \right) \right)}{\norm{\intd \gammabf}} \underbrace{\Tbf(s)}_{\text{Unit tangent at } s}
        \label{equ: Curve Gradient Operator}
    \end{equation}
    This encapsulates the tangential derivative with the actual direction of the tangent vector.
\end{definition}

Similarly, for Laplacian,
\begin{definition}[Second Derivative Laplacian Operator ``Curve Laplacian'']
    Define second derivative operator $\curvelaplacian$ in a way that for $f: M \rightarrow \mathbb{R}$,
    \begin{equation}
        \curvelaplacian_{\gammabf} u \left( \gammabf (s) \right) = \curvenabla_{\gammabf}^{T} \curvenabla_{\gammabf} u \left( \gammabf (s) \right)
    \end{equation}
\end{definition}


\end{document}
