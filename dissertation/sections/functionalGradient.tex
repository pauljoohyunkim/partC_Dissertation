\documentclass[../dissertation.tex]{subfiles}

\begin{document}
In order to understand the gradient of a functional, it helps to recall the gradient of a function.
Gradient of a function $f:\mathbb{R}^n \rightarrow \mathbb{R}$ characterises the first-order variation in a certain direction,
that is,
\begin{equation}
    \forall \xbf, \ybf \in \mathbb{R}^n \; \; \;
    \nabla f \left( \xbf \right) \cdot \ybf = \frac{\partial}{\partial \epsilon} f\left( \xbf + \epsilon \ybf \right)|_{\epsilon=0}
\end{equation}
While this is not a conventional definition of function gradient, it is still an equivalent definition.

One could analogously construct the definition of gradient of a functional.
\begin{definition}[Gradient of Functional]
    For a functional $F:\mathcal{X} \rightarrow \mathbb{R}$, define functional gradient $\grad_{\mathcal{X}} F (f)$ as:
    \begin{equation}
        \forall f, g \in \mathcal{X} \; \; \inner{\grad_{\mathcal{X}} F (f), g}_{\mathcal{X}} = \frac{\partial}{\partial \epsilon} F \left( f + \epsilon g \right)|_{\epsilon = 0}
    \end{equation}
    where $\inner{\cdot, \cdot}_{\mathcal{X}}$ is the inner product defined over the inner product function space $\mathcal{X}$.
\end{definition}

Common inner product spaces include $L^2$, $H^1$, which are defined in Appendix \ref{sct: Definitions of Important Inner Product Spaces}.

\begin{remark}
    By assuming time-independence of $f$ in the gradient flow equation (\ref{equ: Gradient Flow Equation}), we get a stationary state equation $\grad_{\mathcal{X}} F(f) = 0$,
    which implies $\frac{\partial}{\partial \epsilon} F \left( f + \epsilon g \right)|_{\epsilon=0} = 0$ for all $g \in \mathcal{X}$.
    This shows that the stationary state of the gradient flow equation is precisely the solution to the Euler-Lagrange equation.
\end{remark}

\subsubsection{Gradient on Integer Sobolev Spaces}
Assume $\Omega$ is a boundary-free domain.
Given $L^2(\Omega) = H^0(\Omega)$ gradient ($\Omega$ will be omitted in this section when unambiguous), one could express Sobolev gradients of other orders.
For functional $F$, write $h \coloneqq \grad_{L^2} F$ for gradient of $F$ in $L^2$, that is, by definition
\begin{equation}
    \frac{\partial}{\partial \epsilon} F \left( f + \epsilon g \right)|_{\epsilon = 0} = \inner{h, g}_{L^2} \eqqcolon \mathcal{D}
    \label{equ: L2 Gradient Explicit Form}
\end{equation}
By integration by parts (IBP) on boundary-free domain $\Omega$,
\footnote{A more careful treatment of IBP is outlined in appendix \ref{sct: Integration By Parts}.}
\begin{align*}
    \mathcal{D} &= \inner{\laplacian \laplacian^{-1} h, g}_{L^2} \\
    &\stackrel{\text{IBP}}{=} \inner{-\nabla \laplacian^{-1} h, \nabla g}_{L^2} \\
    &= \inner{-\laplacian^{-1} h, g}_{H^1}
\end{align*}
So one concludes that
\begin{equation}
    \grad_{H^1} F = -\laplacian^{-1} h
    \label{equ: H1 Gradient}
\end{equation}
Similarly, one could acquire gradient in $H^2$ by
\begin{align*}
    \mathcal{D} &= \inner{\laplacian^{-1} h, \laplacian g}_{L^2} \\
    &= \inner{\laplacian \left( \laplacian^{-2} h \right), \laplacian g}_{L^2} \\
    &= \inner{\laplacian^{-2} h, g}_{H^2}
\end{align*}
Hence,
\begin{equation}
    \grad_{H^2} F = \laplacian^{-2} h
    \label{equ: H2 Gradient}
\end{equation}
One could take it even further and define gradient in $H^{-1}$,
\begin{align*}
    \mathcal{D} &= \inner{h, \laplacian \laplacian^{-1} g}_{L^2} \\
    &= \inner{-\nabla h, \nabla \left( \laplacian^{-1} g \right)}_{L^2} \\
    &= \inner{-\laplacian h, g}_{H^{-1}}
\end{align*}
So,
\begin{equation}
    \grad_{H^{-1}} F = -\laplacian h
    \label{equ: H-1 Gradient}
\end{equation}
\begin{remark}
    From (\ref{equ: H1 Gradient}), (\ref{equ: H2 Gradient}), and (\ref{equ: H-1 Gradient}), one may deduce that
    choosing the right Sobolev space for a functional may make the gradient flow equation much easier to solve.
    For example, given functional $F (f) \coloneqq \int_{\Omega} |\nabla f (x)|^2 \intd V$ over an open set $\Omega$,
    $L^2$ gradient turns out to be $\grad_{L^2} F (f) = h = -\laplacian f$. 
    Instead of solving $L^2$ gradient flow equation $\frac{\partial}{\partial t} f = \laplacian f$ (Refer to Figure \ref{fig: GF} for the case $\Omega = (-1, 1)$.),
    solving $H^1$ gradient flow equation $\frac{\partial}{\partial t}f = -f$ is more trivial
    as it practically collapses down to solving an ordinary differential equation.
\end{remark}

\end{document}
