\documentclass[../dissertation.tex]{subfiles}

\begin{document}
\subsection{Motivating Example: Heat Equation}
Suppose one needs to solve the heat equation $\frac{\partial u}{\partial t} = \frac{\partial^2 u}{\partial x^2}$ for $u = u(x,t)$ over $\left[ -L, L \right] \times \left[ 0, T_f \right]$
with initial data $u\left( x,0 \right) = u_0 (x)$
and Neumann boundary condition $u'(-L) = u'(L) = 0$.

One could diescretise the problem by constructing a mesh;
taking equispaced points on the spatial domain $\left[ -L,L \right]$
and the time domain $\left[ 0,T_f \right]$.
\begin{align*}
    X_i &= -L + i \left( \Delta X \right) & \text{for } i = 0, 1, \cdots, N \\
    T_j &= j \left( \Delta T \right) & \text{for } j = 0, 1, \cdots, M
\end{align*}
where $N, M$ are the resolution of respective domains,
and $\Delta X \coloneqq \frac{2L}{N}$ and $\Delta T \coloneqq \frac{T_f}{M}$.

Now the idea is to approximate the initial boundary value problem by the following algebraic (difference) equations.
\begin{align}
    \mathcal{D}_t U_{i}^j &= \mathcal{D}_x^+ \mathcal{D}_x^- U_{i}^j 
    \label{equ: Discretised Heat Equation}
    \\
    U_{i}^0 &= u\left( X_i \right) \\
    U_{0}^j - U_{-1}^j &= 0
    \label{equ: Discretised Neumann BC 1}
    \\
    U_{N+1}^j - U_{N}^j &= 0
    \label{equ: Discretised Neumann BC 2}
\end{align}
for $i=0,1,\cdots,N$ and $j = 0, 1, \cdots, M$,
where $\mathcal{D}_z$ is a finite difference operator that approximates a derivative in $z$ variable ($t$ for time, $x$ for spatial variable).
Note that ``virtual points'' $X_{-1}$ and $X_{N+1}$ are added for discretised Neumann boundary condition (\ref{equ: Discretised Neumann BC 1}) and (\ref{equ: Discretised Neumann BC 2}).
One interprets $U_i^j$ as an approximation of $u \left( X_i, T_j \right)$.

\subsubsection{First-Order Finite Difference Operator}
For utilising finite difference method for PDEs, one must choose an operator $\mathcal{D}_z$ to approximate the derivative with respect to variable $z$.

Consider the \textbf{forward difference operator} to get an approximation of $f'(z)$ at $z = Z_i$:
\begin{equation}
    \mathcal{D}_z^+ f \left( Z_i \right) = \frac{f \left( Z_{i+1} \right) - f\left( Z_i \right)}{\Delta Z} = f'\left( Z_i \right) + \underbrace{O \left( \Delta Z \right)}_{\text{Consistency Error}}
\end{equation}

Consider the \textbf{backward difference operator}, another approximation of $f'(z)$ at $z = Z_i$:
\begin{equation}
    \mathcal{D}_z^- f \left( Z_i \right) = \frac{f \left( Z_i \right) - f\left( Z_{i-1} \right)}{\Delta Z} = f'\left( Z_i \right) + \underbrace{O \left( \Delta Z \right)}_{\text{Consistency Error}}
\end{equation}

One can combine forward difference and backward difference to formulate the \textbf{central difference operator}, again for approximation of $f'(z)$ at $z = Z_i$:
\begin{equation}
    \mathcal{D}_z^\circ f \left( Z_i \right) = \frac{f \left( Z_{i+1} \right) - f\left( Z_{i-1} \right)}{2 \Delta Z} = f'\left( Z_i \right) + \underbrace{O \left( \left( \Delta Z \right)^2 \right)}_{\text{Consistency Error}}
\end{equation}
Note that one could write $\mathcal{D}_z^\circ f \left( Z_i \right) = \frac{1}{2}\left( D_z^+ + D_z^- \right) f \left( Z_i \right) $

While central difference operator may seem like the best choice in terms of consistency error,
different choice of operators may result in much easier problem to solve albeit at the cost of increased consistency error.

\subsubsection{Second-Order Finite Difference Operator}
For PDEs involving second derivatives (such as the Laplace's equation), one must also be able to approximate the second derivatives with respect to variable $z$.
A sensible choice is to use both the forward difference operator and and the backward difference operator, known as the \textbf{second divided difference operator}:
\begin{equation}
    \mathcal{D}_z^+ \mathcal{D}_z^- f \left( Z_i \right) = \frac{f \left( Z_{i+1} \right) - 2 f \left( Z_i \right) + f \left( Z_{i-1} \right)}{\left( \Delta Z \right)^2} = f''\left( Z_i \right) + \underbrace{O \left( \left( \Delta Z \right)^2 \right)}_{\text{Consistency Error}}
\end{equation}
One can easily check that $\mathcal{D}_z^+ \mathcal{D}_z^- = \mathcal{D}_z^- \mathcal{D}_z^+$.

\subsection{Euler Schemes for Heat Equation}
\label{sct: Euler Schemes for Heat Equation}
Different choices for the first-order operator in (\ref{equ: Discretised Heat Equation}) for the time variable leads to different equations to solve.

Taking $\mathcal{D}_t = \mathcal{D}_t^+$, one acquires the \textbf{explicit Euler scheme}:
\begin{equation}
    \frac{U_{i}^{j+1} - U_{i}^j}{\Delta T} = \frac{U_{i+1}^j - 2 U_{i}^{j} + U_{i-1}^j}{\left( \Delta X \right)^2}
\end{equation}
Note that figure \ref{fig: GF} is the numerical solution acquired using this scheme.
While it is easy to compute the ``next time step'' by observing that this can be written as $U_{i}^{j+1} = U_{i}^j + \frac{\Delta T}{\left( \Delta X \right)^2} \left( U_{i+1}^j - 2 U_{i}^j + U_{i-1}^{j} \right)$
(where the RHS is known),
a formal analysis using discrete Fourier transform shows that this scheme is stable if and only if $\mu \coloneqq \frac{\Delta T}{\left( \Delta X \right)^2} \leq \frac{1}{2}$.

Conversely, taking $\mathcal{D}_t = \mathcal{D}_t^-$, one acquires the \textbf{implicit Euler scheme}:
\begin{equation}
    \frac{U_{i}^{j} - U_{i}^{j-1}}{\Delta T} = \frac{U_{i+1}^j - 2 U_{i}^{j} + U_{i-1}^j}{\left( \Delta X \right)^2}
\end{equation}
which is unconditionally stable, but requires solving a nonlinear equation to acquire the next step.
For more detail, see \cite{nspde}.



\end{document}
