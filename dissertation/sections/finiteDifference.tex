\documentclass[../dissertation.tex]{subfiles}

\begin{document}
\subsection{Motivating Example: Heat Equation}
Suppose one needs to solve the heat equation $\frac{\partial u}{\partial t} = \frac{\partial^2 u}{\partial x^2}$ for $u = u(x,t)$ over $\left[ 0, L \right] \times \left[ 0, T_f \right]$
with initial data $u\left( x,0 \right) = u_0 (x)$.

One could diescretise the problem by constructing a mesh;
taking equispaced points on the spatial domain $\left[ 0,L \right]$
and the time domain $\left[ 0,T_f \right]$.
\begin{align*}
    X_i &= i \left( \Delta X \right) & \text{for } i = 0, 1, \cdots, N \\
    T_j &= j \left( \Delta T \right) & \text{for } j = 0, 1, \cdots, M
\end{align*}
where $N, M$ are the resolution of respective domains,
and $\Delta X \coloneqq \frac{L}{N}$ and $\Delta T \coloneqq \frac{T_f}{M}$.

Now the idea is to approximate the actual solution $u$ at $\left( x,t \right) = \left( X_i, T_j \right)$ of the heat equation by the following algebraic (difference) equations.
\begin{align}
    \mathcal{D}_t U_{i}^j &= \mathcal{D}_x^+ \mathcal{D}_x^- U_{i}^j \\
    U_{i}^0 &= u\left( X_i \right)
\end{align}
for $i=0,1,\cdots,N$ and $j = 0, 1, \cdots, M$,
where $\mathcal{D}_z$ is a finite difference operator that approximates a derivative in $z$ variable ($t$ for time, $x$ for spatial variable).
One interprets $U_i^j$ as an approximation of $u \left( X_i, T_j \right)$.

\subsubsection{First-Order Finite Difference Operator}
For utilising finite difference method for PDEs, one must choose an operator $\mathcal{D}_z$ to approximate the derivative with respect to variable $z$.

Consider the \textbf{forward difference operator} to get an approximation of $f'(z)$ at $z = Z_i$:
\begin{equation}
    \mathcal{D}_z^+ f \left( Z_i \right) = \frac{f \left( Z_{i+1} \right) - f\left( Z_i \right)}{\Delta Z} = f'\left( Z_i \right) + \underbrace{O \left( \Delta Z \right)}_{\text{Consistency Error}}
\end{equation}

Consider the \textbf{backward difference operator}, another approximation of $f'(z)$ at $z = Z_i$:
\begin{equation}
    \mathcal{D}_z^- f \left( Z_i \right) = \frac{f \left( Z_i \right) - f\left( Z_{i-1} \right)}{\Delta Z} = f'\left( Z_i \right) + \underbrace{O \left( \Delta Z \right)}_{\text{Consistency Error}}
\end{equation}

One can combine the idea of forward difference and backward difference to motivate the \textbf{central difference operator}, again for approximation of $f'(z)$ at $z = Z_i$:
\begin{equation}
    \mathcal{D}_z^\circ f \left( Z_i \right) = \frac{f \left( Z_{i+1} \right) - f\left( Z_{i-1} \right)}{2 \Delta Z} = f'\left( Z_i \right) + \underbrace{O \left( \left( \Delta Z \right)^2 \right)}_{\text{Consistency Error}}
\end{equation}
Note that one could write $\mathcal{D}_z^\circ f \left( Z_i \right) = \frac{1}{2}\left( D_z^+ + D_z^- \right) f \left( Z_i \right) $

While central difference operator may seem like the best choice in terms of consistency error,
different choice of operators may result in much easier problem to solve albeit the cost of increased consistency error.

\subsubsection{Second-Order Finite Difference Operator}
For PDEs involving second derivatives (such as the Laplace's equation), one must also be able to approximate the second derivatives with respect to variable $z$.
A sensible choice is to use both the forward difference operator and and the backward difference operator, known as the \textbf{second divided difference operator}:
\begin{equation}
    \mathcal{D}_z^+ \mathcal{D}_z^- f \left( Z_i \right) = \frac{f \left( Z_{i+1} \right) - 2 f \left( Z_i \right) + f \left( Z_{i-1} \right)}{\left( \Delta Z \right)^2} = f''\left( Z_i \right) + \underbrace{O \left( \left( \Delta Z \right)^2 \right)}_{\text{Consistency Error}}
\end{equation}
One can easily check that $\mathcal{D}_z^+ \mathcal{D}_z^- = \mathcal{D}_z^- \mathcal{D}_z^+$.

\end{document}
