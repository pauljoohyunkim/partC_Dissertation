\documentclass{beamer}
\usetheme{Madrid}

\AtBeginSection[]{
  \begin{frame}
  \vfill
  \centering
  \begin{beamercolorbox}[sep=8pt,center,shadow=true,rounded=true]{title}
    \usebeamerfont{title}\insertsectionhead\par%
  \end{beamercolorbox}
  \vfill
  \end{frame}
}


\usepackage{amsmath}
\usepackage{amssymb}
\usepackage{mathtools}
\usepackage[]{graphicx}
\usepackage{libertinus}
\usepackage{caption}
\usepackage{subcaption}
\usepackage[backend=biber]{biblatex}
\addbibresource{bibliography.bib}

\newcommand{\gammabf}{\boldsymbol{\gamma}}
\newcommand{\intd}{\, \text{d}}
\newcommand{\norm}[1]{\lvert \lvert #1 \rvert \rvert}


\title{Untangling Knots Through Curve Repulsion}
\titlegraphic{\includegraphics[height=2cm]{oxfordlogo.eps}}
\author{Joo-Hyun Paul Kim}

\begin{document}

% Title Page
\begin{frame}
    \titlepage
\end{frame}

\begin{frame}
    \frametitle{What the curious folks ponder about}
    \tableofcontents
\end{frame}

\section{Introduction}
\begin{frame}
    \frametitle{A Cool Knot}
    \begin{figure}[h]
        \centering
        \includegraphics[scale=0.25]{knot}
        \caption{Imagine your earphones getting tangled like this\ldots}
    \end{figure}
\end{frame}


\begin{frame}
    \frametitle{Aim}
    \begin{itemize}
        \item<1-> Finding a ``homotopy'' from a knot to an unknot.
            \begin{figure}[h]
                \centering
                \includegraphics[scale=0.2]{knotsolving}
                \caption{Unknots of test knots.\cite{YSC2021}}
            \end{figure}
        \item<2-> ``Avoiding self-intersection''
    \end{itemize}
\end{frame}

\begin{frame}
    \frametitle{General Strategy}
    \begin{enumerate}
        \item<1-> Define curve energy; penalizing the closeness of points on a curve.
            \begin{itemize}
                \item Closeness of points on curve is a natural characteristic of a tangled curve.
            \end{itemize}
        \item<2-> Attempt to decrease the curve energy by continuously deforming the curve.
            \begin{itemize}
                \item We evolve the curve according to the gradient flow equation.
                \item There is a freedom in choosing the ``gradient'' here.
            \end{itemize}
        \item<3-> We expect the stationary state to be the ``unknot''
            \begin{itemize}
                \item Or at least a simpler state\ldots
            \end{itemize}
    \end{enumerate}

\end{frame}

\section{Tangent-Point Energy}
\begin{frame}
    \frametitle{Defining Curve Energy}
    \onslide<1->{
        Given an (arc-length parameterised) curve $\gammabf:M \rightarrow \mathbb{R}^3$, we wish to assign energy of the form:
        \begin{equation}
            \mathcal{E} \left( \gammabf \right) \coloneqq \iint_{M^2} k \left( \gammabf_x, \gammabf_y \right) \intd \gamma_x \intd \gamma_y
        \end{equation}
        such that
        \begin{itemize}
            \item $\mathcal{E}$ is high when two non-neighbouring points are close.
        \end{itemize}
    }
    \onslide<2>{
        A na\"ive choice is $k \left( \gammabf_x, \gammabf_y \right) \coloneqq \frac{1}{\norm{\gammabf_x - \gammabf_y}}$
    }
\end{frame}

\begin{frame}
    \frametitle{Pitfall of the ``Simple Energy''}
        \begin{equation*}
            \mathcal{E} \left( \gammabf \right) \coloneqq \iint_{M^2} \frac{1}{\norm{\gammabf_x - \gammabf_y}} \intd \gamma_x \intd \gamma_y
        \end{equation*}
        \begin{figure}[h]
            \centering
            \begin{subfigure}[b]{0.45\textwidth}
                \centering
                \includegraphics[width=\textwidth]{simple-0}
            \end{subfigure}
            \begin{subfigure}[b]{0.45\textwidth}
                \centering
                \includegraphics[width=\textwidth]{simple-1}
            \end{subfigure}
        \end{figure}
\end{frame}

\begin{frame}
    \frametitle{Pitfall of the ``Simple Energy''}
        \begin{equation*}
            \mathcal{E} \left( \gammabf \right) \coloneqq \iint_{M^2} \frac{1}{\norm{\gammabf_x - \gammabf_y}} \intd \gamma_x \intd \gamma_y
        \end{equation*}
        \begin{figure}[h]
            \centering
            \begin{subfigure}[b]{0.45\textwidth}
                \centering
                \includegraphics[width=\textwidth]{simple-2}
            \end{subfigure}
            \begin{subfigure}[b]{0.45\textwidth}
                \centering
                \includegraphics[width=\textwidth]{simple-3}
            \end{subfigure}
        \end{figure}
        \onslide<2->{
            This energy is not well-defined for a lot of curves!
        }
\end{frame}

\begin{frame}
    \frametitle{Buck-Orloff Tangent-Point Energy}
    \begin{itemize}
        \item From the simple energy, need a way to eliminate the contribution of the ``singularity''.
    \end{itemize}
    \begin{definition}[Buck-Orloff Tangent-Point Energy]<2->
        For a smooth curve $\gammabf$, define
        \begin{equation*}
            \mathcal{E} \left( \gammabf \right) \coloneqq \iint_{M^2} k_{4}^2 \left( \gammabf_x, \gammabf_y, \mathbf{T}_x \right) \intd \gamma_x \intd \gamma_y
        \end{equation*}
        where $\mathbf{T}_x$ is the unit tangent vector at $\gammabf_x$,
        with the kernel defined as
        \begin{equation*}
            k_4^2 \left( \mathbf{p}, \mathbf{q}, \mathbf{T} \right) \coloneqq \frac{\norm{\mathbf{T} \wedge \left( \mathbf{p} - \mathbf{q} \right)}^{2}}{\norm{\mathbf{p} - \mathbf{q}}^{4}}
        \end{equation*}
        as \textbf{Buck-Orloff Tangent-Point Energy}.\cite{BO1995}
    \end{definition}
\end{frame}

\begin{frame}
    \frametitle{Intuition}
    
\end{frame}

\begin{frame}
    \frametitle{Bibliography}
    \printbibliography
\end{frame}

\end{document}
