%        File: SimpleEnergy.tex
%     Created: Sat Feb 04 02:00 PM 2023 G
% Last Change: Sat Feb 04 02:00 PM 2023 G
%
\documentclass[a4paper]{article}

\usepackage[]{amsmath}
\usepackage{mathtools}
\usepackage{amsthm}

\newcommand{\dgamma}{\, \text{d}\gamma}

\theoremstyle{definition}
\newtheorem{definition}{Definition}

\title{Simple Energy}
\author{Paul Kim}

\begin{document}
\maketitle
When tangent-point energy was used evolve the curve, it turned out that in a discrete setting,
it is minimized when the points are colinear.

Instead, it might be worth investigating the simplest of curve energies.

\textbf{NOTE: The ``simple energy'' is not scale invariant $\alpha < 2$, which results in the phenomenon of ``reduction to singularity'' when minimizing the energy}.
\section{Simple Energy}
For a closed curve $C = \gamma(t)$ over $t \in M$, define \textbf{simple energy} as:
\begin{equation}
    \mathcal{E}^{\alpha} \left( \gamma \right) \coloneqq \iint_{M^2} \frac{1}{|\gamma(x) - \gamma(y)|^{\alpha}} \dgamma_x \dgamma_y
\end{equation}
where $\alpha > 0$.

For $\alpha \geq 1$, this energy is ill-defined in an analytic framework.
However, in a discrete scheme inspired by this, there is a natural way to make this well-defined.

Note that this energy is scale-invariant if and only if $\alpha = 2$. This may be an important fact.
When $\alpha < 2$, this energy scales as $k^{2-\alpha}$ where $k$ is a scale factor to a given curve (eg. $\Gamma = k \gamma$),
meaning that the simple energy is minimized when the curve reduces to singularity (which is not relevant)
On the other hand, if $\alpha > 2$, then by the same reason, energy reduces as the curve expands.

For our purposes, if simple energy is to be used, \underline{$\alpha$ needs to be taken at least 2}.

\section{Discrete Simple Energy}
\begin{definition}
    Given a closed, non-intersection polygonal curve $\Gamma \coloneqq \left( x_0, x_1, \cdots, x_{J-1} \right)$
    ($x_0 = x_J$)
    define \textbf{discrete simple energy of first kind} as:
    \begin{equation}
        E_{1}^{\alpha}\left( \Gamma \right) = \sum_{i=0}^{J-1} \sum_{j \neq i} \frac{1}{|x_i -x_j|^{\alpha}} |x_i - x_{i+1}| |x_j - x_{j+1}|
    \end{equation}
    where $\alpha > 0$.
\end{definition}

This is the simplest energy to compute, but it may not be symmetric.

To address the issue of non-symmetricness, a simple modification can be done:
\begin{definition}
    Given a closed, non-intersection polygonal curve $\Gamma \coloneqq \left( x_0, x_1, \cdots, x_{J-1} \right)$
    ($x_0 = x_J$)
    define \textbf{discrete simple energy of second kind} as:
    \begin{equation}
        E_{2}^{\alpha}\left( \Gamma \right) = \sum_{i=0}^{J-1} \sum_{|i - j| > 1} k_{i,j}^{\alpha} |x_i - x_{i+1}| |x_j - x_{j+1}|
    \end{equation}
    where
    \begin{equation}
        k_{i,j}^{\alpha} = \frac{1}{4} \left( \frac{1}{|x_i - x_j|^{\alpha}} + \frac{1}{|x_{i} - x_{j+1}|^{\alpha}} + \frac{1}{|x_{i+1} - x_{j}|^{\alpha}} + \frac{1}{|x_{i+1} - x_{j+1}|^{\alpha}} \right)
    \end{equation}
    and $\alpha > 0$.
    Note that the indexing must be done cyclically, so $|i-j| > 1$ is modulo $J$.
\end{definition}

\end{document}


