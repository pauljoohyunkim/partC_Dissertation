%        File: SimpleEnergy.tex
%     Created: Sat Feb 04 02:00 PM 2023 G
% Last Change: Sat Feb 04 02:00 PM 2023 G
%
\documentclass[a4paper]{article}

\usepackage[]{amsmath}
\usepackage{mathtools}

\newcommand{\dgamma}{\, \text{d}\gamma}

\title{Simple Energy}
\author{Paul Kim}

\begin{document}
\maketitle
When tangent-point energy was used evolve the curve, it turned out that in a discrete setting,
it is minimized when the points are colinear.

Instead, it might be worth investigating the simplest of curve energies.
\section{Simple Energy}
For a closed curve $C = \gamma(t)$ over $t \in M$, define \textbf{simple energy} as:
\begin{equation}
    \mathcal{E}^{\alpha} \left( \gamma \right) \coloneqq \iint_{M^2} \frac{1}{|\gamma(x) - \gamma(y)|^{\alpha}} \dgamma_x \dgamma_y
\end{equation}
where $\alpha > 0$.

For $\alpha \geq 1$, this energy is ill-defined in an analytic framework.
However, in a discrete scheme inspired by this, there is a natural way to make this well-defined.

\section{Discrete Simple Energy}
Given a closed, non-intersection polygonal curve $\Gamma \coloneqq \left( x_0, x_1, \cdots, x_{J-1} \right)$
($x_0 = x_J$)
define \textbf{discrete simple energy} as:
\begin{equation}
    E^{\alpha}\left( \Gamma \right) = \sum_{i=0}^{J-1} \sum_{j \neq i} \frac{1}{|x_i -x_j|^{\alpha}} |x_i - x_{i+1}| |x_j - x_{j+1}|
\end{equation}

\end{document}


