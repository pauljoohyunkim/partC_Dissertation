%        File: CurveRepulsion1.tex
%     Created: Fri Jan 20 05:00 PM 2023 G
% Last Change: Fri Jan 20 05:00 PM 2023 G
%
\documentclass[a4paper, 11pt]{article}

\usepackage[]{amsmath}
\usepackage{amssymb}
\usepackage{mathtools}
\usepackage{amsthm}
\usepackage{graphicx}
\graphicspath{ {./images/} }

\newcommand{\x}{\mathbf{x}}
\newcommand{\dx}{\,\text{d}x}
\newcommand{\dy}{\,\text{d}y}
\newcommand{\norm}[1]{||#1||}
\newcommand{\inner}[1]{\langle \langle #1 \rangle \rangle}

\DeclareMathOperator{\grad}{grad}

\theoremstyle{definition}
\newtheorem{definition}{Definition}


\title{Curve Repulsion - 1}
\date{20/01/2023}
\author{Paul Kim}

\begin{document}
\maketitle

\section{Theory behind Discretization}
\begin{figure}[h]
    \centering
    \includegraphics[scale=0.5]{discretization.png}
    \caption{Discretization process}
\end{figure}
We now consider a discretization of a curve.
Index the points on the curve by $\mathcal{I} = \left\{ 1, 2, \cdots,  M \right\}$, such that
points are given by $\left\{ \gamma_1, \gamma_2, \cdots, \gamma_M \right\}$

Using the similar notation to the paper by Yu, Schumacher, and Crane,
for edge $I = \left\{ \gamma_i, \gamma_j \right\} \in E$
and for function $u: \mathbb{R}^3 \rightarrow \mathbb{R}$
\begin{itemize}
    \item $l_I \coloneqq | \gamma_i - \gamma_j |$
    \item $T_I \coloneqq \frac{\gamma_j - \gamma_i}{l_I}$
    \item $\x_I \coloneqq \frac{\gamma_i + \gamma_j}{2}$
    \item $u_I \coloneqq \frac{u_i + u_j}{2}$
        \begin{itemize}
            \item Syntactic sugar: $u_i \equiv u\left( \gamma_i \right)$
        \end{itemize}
    \item $u[I] \coloneqq 
        \begin{pmatrix}
            u_i \\
            u_j
        \end{pmatrix}
        $
\end{itemize}

\subsection{Discrete Energy}
The na\"ive discretization of $\mathcal{E}_{\beta}^\alpha \coloneqq \iint_{M^2} k_{\beta}^{\alpha} \left( \gamma \left( x \right), \gamma \left( y \right), T\left( x \right) \right) \dx_{\gamma} \dy_{\gamma}$
where $k_{\beta}^{\alpha} \left( p, q, T \right) \coloneqq \frac{|T \times \left( p-q \right)|^\alpha}{|p-q|^{\beta}}$
is given by
\begin{equation}
    \sum_{I \in E} \sum_{J \in E} \int_{\bar I} \int_{\bar J} k_{\beta}^{\alpha} \left( \gamma (x), \gamma (y), T_I \right) \dx_{\gamma} \dy_{\gamma}
    \label{equ: Ill-defined Discrete Energy}
\end{equation}
However, in a polygonal curve (hence the discretized curve), (\ref{equ: Ill-defined Discrete Energy}) is ill-defined.
\begin{figure}[h]
    \centering
    \includegraphics[scale=0.5]{polygon.png}
    \caption{Near each vertex, the integrand is unbounded.}
\end{figure}

So resolve this by removing the two neighboring edges.\footnote{In the limit, the contribution from this
removed edge goes to zero.}
Also approximate the kernel by the average of the kernel evaluated at each pair of appropriate edges (total: 4)
\begin{align}
    \hat{\mathcal{E}}_{\beta}^{\alpha} &\coloneqq \sum_{I, J \in E, I \cap J = \emptyset} \left( \hat{k}_{\beta}^\alpha \right)_{I,J} l_I l_J \\
    \left( \hat{k}_{\beta}^{\alpha} \right)_{I,J} &\coloneqq \frac{1}{4} \sum_{i \in J, j \in J} k_{\beta}^{\alpha} \left( \gamma_i, \gamma_j, T_I \right)
\end{align}

\section{Discrete Gradient Flow in $L^2$ for Closed Loop}
Suppose a curve is discretized as position vectors:
$x_1, x_2, \cdots, x_M$ (and $x_{M+1} \coloneqq x_1$).

Also denote the edge from $x_i$ to $x_{i+1}$ as $I_i$ (as opposed to the previous section).

The \textbf{discretized energy} $E$ can be expressed as:
\begin{align}
    E &= \sum_{i = 1}^{M} \sum_{\substack{j = 1 \\ |j-i| > 1}} k_{i,j} \norm{x_{i+1} - x_i} \, \norm{x_{j+1} - x_j} \\
    k_{i,j} &= \frac{1}{4} \left( 
        k_{\beta}^{\alpha} \left( x_i, x_j, T_i \right)
        + k_{\beta}^{\alpha} \left( x_i, x_{j+1}, T_i \right)
        + k_{\beta}^{\alpha} \left( x_{i+1}, x_j, T_i \right)
        + k_{\beta}^{\alpha} \left( x_{i+1}, x_{j+1}, T_i \right)
    \right)
\end{align}

\begin{figure}[h]
    \centering
    \includegraphics[scale=0.6]{kernel-2x2.png}
    \caption{Kernel $k_{i,j}$ computation}
\end{figure}

Recall the definition of differential, gradient, and gradient flow.
\begin{definition}[Differential]
    Given functional $\mathcal{E}(\gamma)$, the \textbf{differential} is defined as:
    \begin{equation}
        \text{d}\mathcal{E}|_{\gamma} (u) = \lim_{\epsilon \rightarrow 0} \frac{1}{\epsilon} \left( \mathcal{E}\left( \gamma + \epsilon u \right) - \mathcal{E}\left( \gamma \right) \right)
    \end{equation}
\end{definition}

\begin{definition}[Gradient]
    Given functional $\mathcal{E}\left( \gamma \right)$ and space $V$, the \textbf{gradient} $\grad \mathcal{E}$ is the unique function satisfying the following for any function $u$:
    \begin{equation}
        \inner{\grad \mathcal{E}, u}_V = \text{d} \mathcal{E} (u)
    \end{equation}
    Note that the LHS is a inner product of two vector-valued functions.
    A natural inner product in $L^2$ to define is:
    \begin{equation}
        \inner{u,v}_{L^2} \coloneqq \int_{\Omega} u \cdot v \dx
    \end{equation}

\end{definition}

\begin{definition}[Gradient Flow]
    Given functional $\mathcal{E}(\gamma)$, the \textbf{gradient flow} equation is defined as:
    \begin{equation}
        \frac{d}{dt} \gamma = - \grad \mathcal{E} (\gamma)
    \end{equation}
    Note that $\gamma = \left( \gamma_x, \gamma_y, \gamma_z \right)^T \in \mathbb{R}^3$, so it might be clearer to write:
    \begin{equation}
        \frac{d}{dt}
        \begin{pmatrix}
            \gamma_x \\
            \gamma_y \\
            \gamma_z
        \end{pmatrix}
        =
        \grad \mathcal{E} \left( 
            \begin{pmatrix}
                \gamma_x \\
                \gamma_y \\
                \gamma_z
            \end{pmatrix}
        \right)
    \end{equation}
\end{definition}

Gradient flow equation in $L^2$ is given by:
\begin{equation}
    \dot{x}_i = - \frac{\partial E}{\partial x_i}
\end{equation}

\end{document}


