%        File: circle-tangent-point-energy.tex
%     Created: Fri Feb 03 01:00 PM 2023 G
% Last Change: Fri Feb 03 01:00 PM 2023 G
%
\documentclass[a4paper]{article}

\usepackage[]{amsmath}
\usepackage{mathtools}

\newcommand{\dx}{\, \text{d} x}
\newcommand{\dy}{\, \text{d} y}
\newcommand{\ds}{\, \text{d} s}
\newcommand{\dt}{\, \text{d} \theta}
\newcommand{\dphi}{\, \text{d} \varphi}
\newcommand{\rb}{\mathbf{r}}

\title{Tangent-Point Energy of a Circle}
\author{Paul Kim}

\begin{document}
\maketitle

\section{Circle}
Given the tangent point energy from Yu, Crane, Schumacher with $\alpha = 2$, $\beta = 4$
(might be identical to the one to Buck, Orloff version)
\begin{equation}
    \mathcal{E}_{4}^2 \left( \gamma \right) \coloneqq \iint_{M^2} k_{4}^2 \left( \gamma (x), \gamma (y), T(x) \right) \dx_{\gamma} \dy_{\gamma}
    \label{equ: Energy}
\end{equation}
where tangent-point kernel is defined as
\begin{equation}
    k_{4}^2 (p, q, T) \coloneqq \frac{|T \wedge \left( p - q \right)|^2}{|p-q|^\beta}
\end{equation}
one could show that the tangent-point energy of a circle to be $\pi^2$.

Consider parameterizing a circle at the origin with radius $a$ as:
\begin{align}
    \rb_1 &= a \left( \cos \theta, \sin \theta, 0 \right) ^ T \\
    \rb_2 &= a \left( \cos \varphi, \sin \varphi, 0 \right) ^ T
\end{align}
Note we may express $T$ as $T\left( \theta \right) = \left( - \sin \theta, \cos \theta, 0 \right) ^ T$

Then, the tangent point energy is:
\begin{align}
    \iint_{S^1 \times S^1} \frac{|T \wedge \left( \rb_1 - \rb_2 \right)|^2}{|\rb_1 - \rb_2|^4} \ds_1 \ds_2
    &= \int_{\theta = 0}^{2 \pi} \int_{\varphi = 0}^{2 \pi} \frac{|T|^2 |\rb_1 - \rb_2|^2 - \left( T \cdot \left( \rb_1 - \rb_2 \right) \right)^2}{|\rb_1 - \rb_2|^4} a \dt a \dphi \\
    &= \int_{\theta = 0}^{2 \pi} \int_{\varphi = 0}^{2 \pi} \frac{4 a^2 \sin^2 \frac{\theta - \varphi}{2}}{4 a^4 \left( -1 + \cos \left( \theta - \varphi \right) \right)^2} a \dt a \dphi \\
    &= \int_{\theta = 0}^{2 \pi} \int_{\varphi = 0}^{2 \pi} \frac{\sin^2 \frac{\theta - \varphi}{2}}{\left( -1 + \cos \left( \theta - \varphi \right) \right)^2} \dt \dphi \\
    &= \pi^2
\end{align}

Note that this is scale invariant.

\end{document}


