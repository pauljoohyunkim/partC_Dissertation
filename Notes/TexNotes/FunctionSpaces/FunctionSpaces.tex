\documentclass[a4paper]{article}

\usepackage{amssymb}
\usepackage{mathtools}
\usepackage{amsthm}

\theoremstyle{definition}
\newtheorem{definition}{Definition}

\newcommand{\inner}[2]{\langle #1, #2 \rangle}
\newcommand{\dx}{\, \text{d} x}
\newcommand{\dy}{\, \text{d} y}

\title{Functional Analysis: Gradient}
\author{Paul Kim}
\date{\today}

\begin{document}
\maketitle

\section{Gradient on $\mathbb{R}^n$}
For $f:\mathbb{R}^n \rightarrow \mathbb{R}$, we may define the gradient $\nabla f$ as:
\begin{definition}
    $\nabla f \eqqcolon v$ such that
    \begin{equation}
        \frac{\partial}{\partial \epsilon} f\left( x + \epsilon y \right) |_{\epsilon = 0} = \inner{v}{y}_{\mathbb{R}^n}
    \end{equation}
\end{definition}

\section{Gradient on $L^2$}
Similarly, for $E: L^2 \rightarrow \mathbb{R}$ (a functional), we define the gradient $\nabla E$ as:
\begin{definition}
    $\nabla E$ such that
    \begin{equation}
        \frac{\partial}{\partial \epsilon} E\left[ f + \epsilon g \right] | _{\epsilon = 0} = \inner{\nabla E}{g}_{L^2}
    \end{equation}
\end{definition}

\subsection{Dirichlet Energy}
For Dirichlet energy defined by $E(f) = \int_{\mathbb{R}} |\nabla f|^2 \dx$,
\begin{equation}
    \frac{\partial}{\partial \epsilon} E\left( f + \epsilon g \right) |_{\epsilon = 0} = \int_{\mathbb{R}} \frac{\delta E}{\delta f} g
\end{equation}

\subsection{Tangent Point Energy}
For our tangent point energy defined by $E(\gamma) = \int_{M^2} \left( \cdot \right) \dx_{\gamma} \dy_{\gamma}$
\begin{equation}
    \frac{\partial}{\partial \epsilon} E(\gamma + \epsilon \delta) |_{\epsilon = 0} = \int_{M^2} \frac{\delta E}{\delta \gamma} \delta
\end{equation}


\end{document}


