\documentclass[a4paper]{article}

\usepackage[]{amsmath}
\usepackage{amssymb}
\usepackage{mathtools}
\usepackage{amsthm}

\theoremstyle{definition}
\newtheorem{definition}{Definition}

\newcommand{\inner}[2]{\langle #1, #2 \rangle}
\newcommand{\dx}{\, \text{d} x}
\newcommand{\dy}{\, \text{d} y}

\DeclareMathOperator{\grad}{grad}

\title{Functional Analysis: Gradient}
\author{Paul Kim}
\date{\today}

\begin{document}
\maketitle

\section{Gradient on $\mathbb{R}^n$}
For $f:\mathbb{R}^n \rightarrow \mathbb{R}$, we may define the gradient $\nabla f$ as:
\begin{definition}
    $\nabla f \eqqcolon v$ such that
    \begin{equation}
        \frac{\partial}{\partial \epsilon} f\left( x + \epsilon y \right) |_{\epsilon = 0} = \inner{v}{y}_{\mathbb{R}^n}
    \end{equation}
\end{definition}

\section{Gradient on $L^2$}
Similarly, for $E: L^2 \rightarrow \mathbb{R}$ (a functional), we define the gradient $\nabla E$ as:
\begin{definition}
    $\nabla E$ such that
    \begin{equation}
        \frac{\partial}{\partial \epsilon} E\left[ f + \epsilon g \right] | _{\epsilon = 0} = \inner{\nabla E}{g}_{L^2}
    \end{equation}
\end{definition}

\subsection{Dirichlet Energy}
For Dirichlet energy defined by $E(f) = \int_{\mathbb{R}} |\nabla f|^2 \dx$,
\begin{equation}
    \frac{\partial}{\partial \epsilon} E\left( f + \epsilon g \right) |_{\epsilon = 0} = \int_{\mathbb{R}} \frac{\delta E}{\delta f} g
\end{equation}

\subsection{Tangent Point Energy}
For our tangent point energy defined by $E(\gamma) = \int_{M^2} \left( \cdot \right) \dx_{\gamma} \dy_{\gamma}$
\begin{equation}
    \frac{\partial}{\partial \epsilon} E(\gamma + \epsilon \delta) |_{\epsilon = 0} = \int_{M^2} \frac{\delta E}{\delta \gamma} \delta
\end{equation}


\section{Gradient on Integer Sobolev Spaces}
For integer Sobolev spaces $H^k$, inner product is defined by $\inner{f}{g}_{H^k} = \sum_{i=0}^{k} \inner{D^i f}{D^i g}$.
Write the $L^2$ gradient as $h$: $\frac{\partial}{\partial \epsilon} \mathcal{E} (f+\epsilon g) |_{\epsilon = 0} = \inner{h}{g}_{L^2} \eqqcolon \mathcal{D}$
\subsection{Gradient on $H^1$}
\begin{align}
    \mathcal{D} &= \inner{\nabla^2 (\Delta^{-1})h}{g}_{L^2} \\
    &= \inner{-\nabla \left( \Delta^{-1} \right)h }{\nabla g}_{L^2} \\
    &= \inner{- \Delta^{-1} h}{g}_{H^1}
\end{align}
So, $\grad_{H^1} \mathcal{E} = -\Delta^{-1} h$

\subsection{Gradient on $H^{-1}$}
Note that $\inner{f}{g}_{H^{-1}} = \inner{\Delta^{-1} f}{\Delta^{-1} g}_{H^1} = \inner{\nabla \left( \Delta^{-1} f \right)}{\nabla \left( \Delta^{-1} g \right)}_{L^2}$
Now note,
\begin{align}
    \mathcal{D} &= \inner{h}{g}_{L^2} \\
    &= \inner{-\nabla h}{\nabla \left( \Delta^{-1} g \right)}_{L^2} \\
    &\stackrel{f=-\Delta h}{=} \inner{-\Delta h}{g}_{H^{-1}}
\end{align}
So, $\grad_{H^{-1}} \mathcal{E} = -\Delta h$

\subsection{Gradient on $H^2$}
\begin{align}
    \mathcal{D} &= \inner{h}{g}_{L^2} \\
    &= \inner{\Delta^{-1} h}{\Delta g}_{L^2} \\
    &= \inner{\Delta \left( \Delta^{-2} h \right)}{\Delta g}_{L^2} \\
    &= \inner{\Delta^{-2} h}{g}_{H^2}
\end{align}
So, $\grad_{H^2} \mathcal{E} = \Delta^{-2} h$

\end{document}


