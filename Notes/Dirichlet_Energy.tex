%        File: Dirichlet Energy.tex
%     Created: Fri Dec 02 07:00 PM 2022 G
% Last Change: Fri Dec 02 07:00 PM 2022 G
%
\documentclass[a4paper]{article}
\usepackage[]{amsmath}
\usepackage{amsthm}
\usepackage{amssymb}
\usepackage{mathtools}



\newcommand{\dx}{\, \text{d} x}
\newcommand{\dV}{\, \text{d} V}
\newcommand{\ds}{\, \text{d} s}
\newcommand{\dxx}{\, \text{d} \mathbf{x}}


\theoremstyle{definition}
\newtheorem{definition}{Definition}

\DeclareMathOperator{\grad}{grad}

\title{Dirichlet Energy in Different Spaces}
\author{ }

\begin{document}
\maketitle
\section{Relevant Definitions}
\begin{definition}[Gradient Flow]
    Given an energy (functional) $\mathcal{E}(f)$, \emph{gradient flow} is:
    \begin{equation}
        \frac{d}{dt}f = - \grad \mathcal{E}(f)
        \label{equ: Gradient Flow}
    \end{equation}
\end{definition}

\begin{definition}[Differential]
    \emph{Differential} $d\mathcal{E}$ describes change in $\mathcal{E}$ due to\footnote{analogous in traditional vector space would be ``in the direction of'' $u$} $u$:
 f  \begin{equation}
        d \mathcal{E}|_f(u) = \lim_{\epsilon \rightarrow 0} \frac{1}{\epsilon} \left( \mathcal{E}(f+\epsilon u) - \mathcal{E}(f) \right)
        \label{equ: Differential}
    \end{equation}
\end{definition}

\begin{definition}[Gradient]
    Given a space $X$, \emph{gradient} of $\mathcal{E}$ is the unique function $\grad_X \mathcal{E}$ such that,
    \begin{equation}
        \langle \grad_X \mathcal{E}, u \rangle_X = d \mathcal{E}(u) \hspace{1cm} \forall u \in X
        \label{equ: Gradient}
    \end{equation}
\end{definition}

\section{Dirichlet Energy Example}

\begin{definition}[Dirichlet Energy]
    Dirichlet Energy is defined as:
    \begin{equation}
        \mathcal{E}_D(f) \coloneqq \frac{1}{2} \int_{\Omega} |\grad f(x)|^2 \dx
        = \frac{1}{2} \langle \nabla f, \nabla f \rangle_{L^2}
        = -\frac{1}{2} \langle \Delta f, f \rangle_{L^2}
        \label{equ: Dirichlet Energy}
    \end{equation}
    where the last equality comes from IBP.
\end{definition}

Computing the differential $d \mathcal{E}_D|_f (u)$ of Dirichlet energy using (\ref{equ: Differential}),
\begin{align}
    d \mathcal{E}_D|_f(u) &= -\frac{1}{2} \lim_{\epsilon \rightarrow 0} \frac{1}{\epsilon} \left( \langle \Delta (f+\epsilon u), f + \epsilon u \rangle_{L^2} - \langle \Delta f, f \rangle_{L^2} \right) \\
    &= -\frac{1}{2} \lim_{\epsilon \rightarrow 0} \frac{1}{\epsilon} \left( \epsilon \langle \Delta f, u \rangle_{L^2} + \epsilon \langle \Delta u, f \rangle_{L^2} + \epsilon^2 \langle u, u \rangle_{L^2} \right) \\
    &= -\frac{1}{2} (\langle \Delta f,u \rangle_{L^2} + \langle \Delta u, f \rangle_{L^2})
    \label{equ: Dirichlet Differential No IBP}
    \\
    &= -\frac{1}{2} (\langle \Delta f,u \rangle_{L^2} + \langle \Delta f, u \rangle_{L^2})
    \label{equ: Dirichlet Differential with IBP}
    \\
    &= -\langle \Delta f, u \rangle_{L^2}
    \label{equ: Dirichlet Differential}
\end{align}
where the step from (\ref{equ: Dirichlet Differential No IBP}) to (\ref{equ: Dirichlet Differential with IBP}) is by integration by parts \footnote{This trick will be used all over the place when constructing gradients and forming \textbf{natural boundary condition}. See \ref{sct: Natural Boundary Condition} in the Appendix}.

\subsection{Gradient Flow in $L^2$}
By (\ref{equ: Gradient}), we may spot from (\ref{equ: Dirichlet Differential}) that the gradient can be written as
\begin{equation}
    \grad \mathcal{E}_{L^2} = - \Delta f
    \label{equ: L2 Gradient}
\end{equation}
Hence, the gradient flow equation (\ref{equ: Gradient Flow}) can be written as:
\begin{equation}
    \frac{d}{dt} f = \Delta f
    \label{equ: L2 Gradient Flow}
\end{equation}
which is the heat equation.


\subsection{Gradient Flow in $H^1$}
We may express the differential (\ref{equ: Dirichlet Differential}) as
\begin{align}
    d \mathcal{E}_D|_f (u) &= - \langle \Delta f, u \rangle_{L^2} \\
    &= \langle \nabla f, \nabla u \rangle_{L^2} \\
    &= \langle f, u \rangle_{H^1}
    \label{equ: Gradient H1 Derivation}
\end{align}
So by (\ref{equ: Gradient}), the gradient in $H^1$ can be written as
\begin{equation}
    \grad \mathcal{E}_{H^1} = f
    \label{equ: H1 Gradient}
\end{equation}
The gradient flow equation (\ref{equ: Gradient Flow}) can be written as
\begin{equation}
    \frac{d}{dt}f = -f
    \label{equ: H1 Gradient Flow}
\end{equation}
which describes exponential decay.

\subsection{Gradient Flow in $H^{-1}$}
We may express the differential (\ref{equ: Dirichlet Differential}) as
\begin{align}
    d \mathcal{E}_D|_f (u) &= - \langle \Delta f, u \rangle_{L^2} \\
    &= \langle \nabla \left( \Delta f \right), \nabla \left( \Delta^{-1} u \right) \rangle_{L^2} \\
    %&= \langle \nabla f, \nabla u \rangle_{L^2} \\
    %&= \langle \nabla^3 f, \nabla^{-1} u \rangle_{L_2} \\
    &= \langle \Delta f, \Delta^{-1} u \rangle_{H^1} \\
    &= \langle \Delta^2 f, u \rangle_{H^{-1}}
    \label{equ: Gradient H-1 Derivation}
\end{align}
So by (\ref{equ: Gradient}), the gradient in $H^{-1}$ can be written as
\begin{equation}
    \grad \mathcal{E}_{H^{-1}} = \Delta^2 f
    \label{equ: H-1 Gradient}
\end{equation}
The gradient flow equation (\ref{equ: Gradient Flow}) can be written as
\begin{equation}
    \frac{d}{dt}f = - \Delta^2 f
    \label{equ: H-1 Gradient Flow}
\end{equation}

\subsection{Gradient Flow in $H^2$}
We may express the differential (\ref{equ: Dirichlet Differential}) as
\begin{align}
    d \mathcal{E}_D|_f (u) &= - \langle \Delta f, u \rangle_{L^2} \\
    &= -\langle f, \Delta u \rangle_{L^{2}} \\
    &= -\langle \Delta^{-1} f, u \rangle_{H^2}
    \label{equ: Gradient H2 Derivation}
\end{align}
So by (\ref{equ: Gradient}), the gradient in $H^2$ can be written as
\begin{equation}
    \grad \mathcal{E}_{H^2} = - \Delta^{-1} f
    \label{equ: H2 Gradient}
\end{equation}
The gradient flow equation (\ref{equ: Gradient Flow}) can be written as
\begin{equation}
    \frac{d}{dt}f = \Delta^{-1} f
    \label{equ: H2 Gradient Flow}
\end{equation}


\section{Numerically Solving Gradient Flow Equations}
\subsection{Gradient Flow in $L^2$}
For $L^2$, given the gradient flow equation (\ref{equ: L2 Gradient Flow}),
we may write down explicit Euler scheme:
\begin{equation}
    \frac{f_j^{m+1} - f_j^m}{\Delta t} = \frac{f_{j+1}^m - 2 f_{j}^m + f_{j-1}^m}{(\Delta x)^2}
    \label{equ: L2 Numeric}
\end{equation}
which can be rewritten as
\begin{equation}
    f_{j}^{m+1} = f_j^m + \mu \left( f_{j+1}^m - 2 f_{j}^m + f_{j-1}^m \right)
\end{equation}
where $\mu \coloneqq \frac{\Delta t}{(\Delta x)^2}$ is called the \emph{CFL number}. For this explicit Euler scheme to be stable, it is needed that $\mu \leq 0.5$

\subsubsection{Boundary Conditions}
For \emph{periodic boundary condition}, we impose:
\begin{equation*}
    f_0^{m} = f_{J}^m \hspace{1cm} \forall m
\end{equation*}
For \emph{natural boundary condition}, we impose:
\begin{equation*}
    \begin{cases}
        f_{-1}^m = f_{0}^m \\
        f_{J}^m = f_{J+1}^m
    \end{cases}
    \hspace{1cm}
    \forall m
\end{equation*}
where this comes from Neumann BC as derived at subsection \ref{sct: L2 Natural BC}.

\subsubsection{Consistency Error}
This method is known to have consistency error
\begin{equation}
    T_j^m = \mathcal{O}\left( \left( \Delta x \right)^2 + \Delta t \right)
    \label{L2 Consistency Error}
\end{equation}

\subsection{Gradient Flow in $H^1$}
Given the gradient flow equation (\ref{equ: H1 Gradient Flow}),
we may write down explicit Euler scheme:
\begin{equation}
    \frac{f_j^{m+1} - f_j^m}{\Delta t} = -f_{j}^m
    \label{equ: H1 Numeric}
\end{equation}
which can be rewritten as
\begin{equation}
    f_j^{m+1} = f_j^m - (\Delta t)f_j^m
\end{equation}

\subsection{Gradient Flow in $H^{-1}$}
Given the gradient flow equation (\ref{equ: H-1 Gradient Flow}),
we may write down explicit Euler scheme:
\begin{equation}
    \frac{f_j^{m+1} - f_j^m}{\Delta t} = -\frac{f_{j+2}^m - 4 f_{j+1}^m + 6 f_j^m - 4 f_{j-1}^m + f_{j-2}^m}{(\Delta x)^4}
    \label{equ: H-1 Numeric}
\end{equation}

\subsection{Gradient Flow in $H^2$}
Given the gradient flow equation (\ref{equ: H2 Gradient Flow}), it can be rewritten as
\begin{equation}
    \frac{d}{dt} \Delta f = f
\end{equation}
we may write down explicit Euler scheme:
\begin{equation}
    \frac{1}{\Delta t}\left( \frac{f_{j+1}^{m+1} - 2f_j^{m+1} + f_{j-1}^{m+1}}{\left( \Delta x \right)^2} - \frac{f_{j+1}^{m} - 2f_j^{m} + f_{j-1}^{m}}{\left( \Delta x \right)^2} \right) = f_j^m
    \label{equ: H2 Numeric}
\end{equation}
which can be rewritten as
\begin{equation}
    \begin{pmatrix}
        1 & -2 & 1
    \end{pmatrix}
    \begin{pmatrix}
        f_{j-1}^{m+1} \\
        f_{j}^{m+1} \\
        f_{j+1}^{m+1}
    \end{pmatrix}
    =
    \begin{pmatrix}
        1 & -2 + \Delta t (\Delta x)^2 & 1
    \end{pmatrix}
    \begin{pmatrix}
        f_{j-1}^{m} \\
        f_{j}^{m} \\
        f_{j+1}^{m}
    \end{pmatrix}
\end{equation}
Assuming Dirichlet boundary condition $f_0^m = a$ and $f_J^m = b$,
we may write this as a matrix equation:
\begin{align*}
    &\begin{pmatrix}
        -2 & 1 \\
        1 & -2 & 1 \\
        & 1 & -2 & 1 \\
        \vdots \\
        & & & & & 1 & -2 & 1 \\
        & & & & & & 1 & -2
    \end{pmatrix}
    \begin{pmatrix}
        f_1 \\
        f_2 \\
        f_3 \\
        \vdots \\
        f_{J-4} \\
        f_{J-3} \\
        f_{J-2} \\
        f_{J-1}
    \end{pmatrix}^{m+1}
    \\
    &=
    \begin{pmatrix}
        -2 + \Delta t (\Delta x)^2 & 1 \\
        1 & -2 + \Delta t (\Delta x)^2  & 1 \\
        & 1 & -2 + \Delta t (\Delta x)^2 & 1 \\
        \vdots \\
        & & & & & 1 & -2 + \Delta t (\Delta x)^2 & 1 \\
        & & & & & & 1 & -2
    \end{pmatrix}
    \begin{pmatrix}
        f_1 \\
        f_2 \\
        f_3 \\
        \vdots \\
        f_{J-4} \\
        f_{J-3} \\
        f_{J-2} \\
        f_{J-1}
    \end{pmatrix}^{m}
\end{align*}
It might be worth noting that
\begin{equation}
    A_n^{-1} \coloneqq
    \underbrace{
    \begin{pmatrix}
        -2 & 1 & 0 & 0 & \cdots & 0 \\
        1 & -2 & 1 & 0 & \cdots & 0 \\
        0 & 1 & -2 & 1 & \cdots & 0 \\
        \vdots & \vdots & \vdots & \vdots & & \vdots \\
        0 & 0 & \cdots & 1 & -2 & 1 \\
        0 & 0 & \cdots & 0 & 1 & -2
    \end{pmatrix}^{-1}
}_{n \times n}
= -\frac{1}{n+1}
\underbrace{
\begin{pmatrix}
    n & (n-1) & (n-2) & \cdots & 1 \\
    (n-1) & 2(n-1) & 2(n-2) & \cdots & 2 \\
    (n-2) & 2(n-2) & 3(n-2) & \cdots & 3 \\
    \vdots & \vdots & \vdots & \cdots & \vdots \\
    1 & 2 & 3 & \cdots & n
\end{pmatrix}
}_{\text{Symmetric Matrix}}
\end{equation}
In terms of practical computation, it might be worth investigating how the condition number grows as the size of $A_n^{-1}$ (which is equivalent to the condition number of $A_n$) grows.
To do this, we could investigate eigenvalues\footnote{Analytically, solutions to the characteristic equation.}
A (potentially) useful fact about this matrix is that the characteristic equation can be written recursively.
\begin{equation}
    \begin{cases}
        |A_n - \lambda I|\eqqcolon J_n = -\delta J_{n-1} - J_{n-2} \\
        J_1 = -\delta \\
        J_{2} = \delta^2 - 1
    \end{cases}
\end{equation}
where $\delta \coloneqq \lambda + 2$.\footnote{The term \emph{continuant} might be interesting to look at}

This method has consistency error
\begin{equation}
    % See Lab/PDE/Dirichlet Energy/H2_Consistency.nb
    T_j^m = \mathcal{O}\left( \left( \Delta x \right)^2 + \Delta t \right)
    \label{H2 Consistency Error}
\end{equation}


\newpage
\section{Appendix}
\subsection{Natural Boundary Condition}
\label{sct: Natural Boundary Condition}
We pay more attention to the boundary terms in the process of integrating by parts.

Starting from (\ref{equ: Dirichlet Energy}), we compute the differential $d \mathcal{E}_D |_f (u)$ again, but with boundary terms.
Recall that the Dirichlet energy is given by:
\begin{equation*}
    \mathcal{E}_D (f) \coloneqq \frac{1}{2} \int_{\Omega} |\nabla f(x)|^2 \dx = \frac{1}{2} \langle \nabla f, \nabla f \rangle_{L^2}
\end{equation*}

Computing the differential with boundary terms:
\begin{align}
    d \mathcal{E}_D|_f(u) &= \lim_{\epsilon \rightarrow 0}\frac{1}{\epsilon} \left( \mathcal{E}(f+\epsilon u) - \mathcal{E}(f) \right) \\
    &= \frac{1}{2} \lim_{\epsilon \rightarrow 0} \frac{1}{\epsilon} \int_{\Omega} \left( |\nabla f + \epsilon \nabla u|^2 - |\nabla f|^2 \right) \dx \\
    &= \frac{1}{2} \lim_{\epsilon \rightarrow 0} \frac{1}{\epsilon} \int_{\Omega} \left( 2\epsilon \nabla f \cdot \nabla u + \epsilon^2 |\nabla u|^2 \right) \dx \\
    &= \int_{\Omega} \nabla f \cdot \nabla u \dx
    \label{equ: Differential in Integral}
\end{align}

\subsubsection{$L^2$}
\label{sct: L2 Natural BC}
We continue from (\ref{equ: Differential in Integral})
\begin{align}
    d \mathcal{E}_D|_f(u) &= \int_{\Omega} \nabla f \cdot \nabla u \dx \\
    &= \int_{\Omega} \left( \nabla \cdot \left( u \nabla f \right) - u \Delta f \right) \dx \\
    &= \langle -\Delta f, u \rangle_{L^2} + \underbrace{\oint_{\partial \Omega} u \nabla f \cdot \mathbf{n} \ds}_{\text{Boundary Term}}
    \label{equ: L2 Natural Boundary}
\end{align}

So for $L^2$, the we can take the natural boundary condition to be
\begin{equation*}
    \nabla f \cdot \mathbf{n} \equiv 0 \hspace{1cm} \text{on } \partial \Omega
\end{equation*}

\subsubsection{$H^1$}
Note, from (\ref{equ: Differential in Integral}),
\begin{align}
    d \mathcal{E}_D|_f(u) &= \int_{\Omega} \nabla f \cdot \nabla u \dx \\
    &= \langle f, u \rangle_{H^1}
\end{align}

So for $H^1$, there is no need to take a natural boundary condition.
\subsubsection{$H^{-1}$}
We continue from (\ref{equ: L2 Natural Boundary}).
Define $g \coloneqq \Delta f$ and $v \coloneqq \Delta^{-1} u$
\begin{align}
    d \mathcal{E}_D|_f(u) &= \langle -\Delta f, u \rangle_{L^2} + \oint_{\partial \Omega} u \nabla f \cdot \mathbf{n} \ds \\
    &= \langle -g, \Delta v \rangle_{L^2} + \oint_{\partial \Omega} u \nabla f \cdot \mathbf{n} \ds \\
    &= -\int_{\Omega} g \nabla^2 v \dx + \oint_{\partial \Omega} u \nabla f \cdot \mathbf{n} \ds \\
    &= -\int_{\Omega} \left( \nabla \cdot \left( g \nabla v \right) - \nabla g \cdot \nabla v \right)  \dx + \oint_{\partial \Omega} u \nabla f \cdot \mathbf{n} \ds \\
    &= \int_{\Omega} \nabla g \cdot \nabla v \dx -\oint_{\partial \Omega} g \nabla v \cdot \mathbf{n} ds + \oint_{\partial \Omega} u \nabla f \cdot \mathbf{n} \ds \\
    &= \int_{\Omega} \nabla \left( \Delta f \right) \cdot \nabla \left( \Delta^{-1} u \right) \dx 
    + \underbrace{\oint_{\partial \Omega} \left( u \nabla f - (\Delta f) \, \nabla \left( \Delta^{-1} u \right) \right) \cdot \mathbf{n} \ds}_{\text{Boundary Terms}}
\end{align}

\end{document}


